\documentclass[12pt, letterpaper]{article}

\usepackage{tabularx}
\usepackage{booktabs}
\usepackage{fancyhdr}
\usepackage[left=1in, right=1in, top=0.3in, bottom=0.7in]{geometry}

\fancyhead[L]{January 28, 2022}
\fancyhead[C]{3XA3 Problem Statement}
\fancyhead[R]{Group 2}
\pagestyle{fancy}

\title{SE 3XA3: Problem Statement - FlagGenerator}

\author{Group 2: Akram Hannoufa \- hannoufa, James Park \- parkg10, Nathaniel Hu \- hun4
}

\date{January 28, 2022}

% \input{../Comments}

\begin{document}

\begin{table}[hp]
\caption{Revision History} \label{TblRevisionHistory}
\begin{tabularx}{\textwidth}{llX}
\toprule
\textbf{Date} & \textbf{Developer(s)} & \textbf{Change}\\
\midrule
2022-01-25 & Nathaniel Hu & Initial draft of context section created \\
2022-01-27 & Akram Hannoufa & Importance section added \\
2022-01-27 & Ganghoon Park & Introduction section added \\
2022-01-28 & Nathaniel Hu & Minor edits to document date, context section \\
... & ... & ...\\
\bottomrule
\end{tabularx}
\end{table}

\newpage
\maketitle

\section{Introduction}
A flag’s design and colours represent a country along with their beliefs, values, and history. Flags must be unique to each country as it is a form of identity. There are many interesting designs that have bright colours, animals, stars, and other shapes and patterns. Although the design may look simple, there are meanings to every flag. For example, the maple leaf in the Canadian flag symbolizes unity, tolerance, and peace. The red describes Canada’s sacrifice during World War I.

Designing a flag can be difficult as it requires a lot of thought. In addition, it must be unique as it cannot be a copy of another flag. Therefore, the random flag generator can be used to uniquely create personalized random flag based on the input string. In doing so, the user can use the generated flag to represent their people and add meaning to it.


\section{Importance}
In today’s day and age, the importance of uniqueness and self-identity is extremely evident. People are looking for ways to express themselves in a digital manner. With an infinite number of possible strings, and $2^{256}$ possible hashes, a user would be able to find the string that best represents them, and then use their generated image as a way to uniquely represent themselves online.

Additionally, a uniquely generated image can serve as great inspiration to graphic designers and\/or artists. Using a string (or strings) that represent the theme of their art, they can generate an image that was built from these words and use that image, either directly or indirectly, in their work.


\section{Context}
The main (and largest) set of stakeholders for this software will be internet users. These include users of social media websites, players of nation building games that allow custom flags, related forums and instant messaging clients. These stakeholders will make good use of an implementation that solves this problem.

PAGAN is an open-source software that successfully addresses the need for generated avatar graphics that are uniquely generated by a user’s input string (e.g. username, etc.). However, its current documentation is technically-oriented, and not very user friendly for non-technical users. PAGAN also does not have an extensive graphical user interface, which would enhance the software's user-friendliness for non-technical users. The PAGAN software will be run locally on a desktop computer that has a command line and can run Python.

\end{document}