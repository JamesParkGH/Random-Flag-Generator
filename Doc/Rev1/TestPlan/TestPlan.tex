\documentclass[12pt, titlepage]{article}

\usepackage{booktabs}
\usepackage{tabularx}
\usepackage[section]{placeins}
\usepackage{hhline}
\usepackage{lscape}
\usepackage{hyperref}
\usepackage[normalem]{ulem}
\hypersetup{
    colorlinks,
    citecolor=black,
    filecolor=black,
    linkcolor=red,
    urlcolor=blue
}
\usepackage[round]{natbib}

\usepackage{xifthen}
\def\namedlabel#1#2{\begingroup
    #2%
    \def\@currentlabel{#2}%
    \phantomsection\label{#1}\endgroup
}
\newcommand{\newterm}[1]{\label{Term:#1} \MakeUppercase #1}
\newcommand{\term}[2][]{\ifthenelse{\equal{#1}{}}{\hyperref[Term:#2]{\textbf{#2}}}{\hyperref[Term:#1]{\textbf{#2}}}}

\title{SE 3XA3: Test Plan\\Random Flag Generator}

\author{Team \#2, Team Jakriel
		\\ Akram Hannoufa, hannoufa
		\\ Ganghoon (James) Park, parkg10
		\\ Nathaniel Hu, hun4
}

\date{\today}

% \input{../Comments}

\begin{document}

\maketitle

\pagenumbering{roman}
\tableofcontents
\listoftables

\newpage
\begin{table}[h]
\caption{\bf Revision History}
\begin{tabularx}{\textwidth}{p{3cm}p{2cm}X}
\toprule {\bf Date} & {\bf Version} & {\bf Notes}\\
\midrule
March 3, 2022 & 1.0 & Initial Document\\
March 3, 2022 & 1.1 & Added initial draft of Tests for Proof of Concept section\\
March 8/9, 2022 & 1.2 & Made edits to draft of Tests for Proof of Concept section; Added initial draft of System Test Description section\\
March 10, 2022 & 1.3 & Made edits, added to draft of Tests for Proof of
Concept section; Added to draft of System Test Description section\\
March 11, 2022 & 1.4 & Added Comparison to Existing Implementation, Unit Testing Plan and Appendix sections content; made edits\\
\textcolor{red}{April 11/12, 2022} & \textcolor{red}{2.0} & \textcolor{red}{Updated Test Plan documentation for Revision 1 submission}\\
\bottomrule
\end{tabularx}
\end{table}

\newpage

\pagenumbering{arabic}

\noindent This document describes the plan for the software testing of the \textbf{Random Flag Generator} program.

\section{General Information}

\subsection{Purpose}
The purpose of this test plan is to describe the testing and validation
process of The Random Flag Generator \textcolor{red}{(\term[RFG]{RFG})} in
detail. This test plan will properly assess the \textcolor{red}{
\term[FR]{functional}} and \textcolor{red}{\term[NFR]{non-functional
requirements}} of the software as well as its performance and usability. The
\textcolor{red}{\term[Test Case]{test cases}} will address each individual
unit of source code and can be referred to or updated with future
implementations. The test plan aims to minimize the probability of errors
occuring when being run by the \textcolor{red}{\term[User]{user}}.

\subsection{Scope}
The test plan considers and tests all \textcolor{red}{\term[FR]{functional}}
and \textcolor{red}{\term[NFR]{non-functional requirements}} through different
testing procedures. Proof of concept tests are also demonstrated. In addition,
the Random Flag Generator program is compared to the existing \textcolor{red}{
\term[PAGAN]{PAGAN}} implementation, and is discussed to compare its
functionality and improve the design. The unit testing plan for internal
functions and output files is also outlined. Finally, the usability survey 
questions will evaluate user experience(s). The test plan will be conducted
accordingly, and the document will be revised as the project's development
progresses.

\newpage
\subsection{Acronyms, Abbreviations, and Symbols}
    
\begin{table}[htbp]
\caption{\textbf{Table of Abbreviations}} \label{abbrev}

\begin{tabularx}{\textwidth}{p{3cm}X}
\toprule
\textbf{Abbreviation} & \textbf{Definition} \\
\midrule
\newterm{FR} & Functional Requirement\\
\hline
\newterm{GUI} & Graphical User Interface\\
\hline
\newterm{LF} & Look and Feel test\\
\hline
\textcolor{red}{\newterm{NFR}} & \textcolor{red}{\term{Non-Functional Requirement}}\\
\hline
\newterm{PAGAN} & \term{Python Avatar Generator for Absolute Nerds}\\
\hline
\newterm{PE} & Performance test\\
\hline
\textcolor{red}{\newterm{RFG}} & \textcolor{red}{\term{Random Flag Generator}}\\
\hline
\newterm{RGB} & Red Green Blue\\
\hline
\newterm{SRS} & Software Requirements Specification\\
\hline
\newterm{UH} & Usability and Humanity test\\
\hline
\newterm{UI} & User Interface\\
\bottomrule
\end{tabularx}

\end{table}

\newpage
\begin{table}[htbp]
\caption{\textbf{Table of Definitions}} \label{def}

\begin{tabularx}{\textwidth}{p{3cm}X}
\toprule
\textbf{Term} & \textbf{Definition}\\
\midrule
\newterm{Gallery} & Collection of previously generated flags.\\
\hline
\newterm{Graphical User Interface} & A form of UI that allows users to use
  electronic devices using interactive graphics.\\
\hline
\newterm{Hashing} & Algorithm that converts input data to a fixed-size value.
  A hashing function usually outputs a string or hexadecimal value.\\
\hline
\newterm{Input String} & The input of type string from the user.\\
\hline
\newterm{Pytest} & Python testing tool that allows testers to write test code
  and create simple and scalable test cases.\\
\hline
\newterm{Python} & The programming language used in this project.\\
\hline
\newterm{Software Requirements Specification} & A document that details what
  the program/software will do and how it will accomplish the expected
  performance/tasks.\\
\hline
\newterm{System/Program} & Collection of instructions or components that tell
  a computer how to operate.\\
\hline
\newterm{Tester} & An individual testing the software via the user interface
  or the code/test cases.\\
\hline
\textcolor{red}{\newterm{Test Case}} & \textcolor{red}{a specification of
  inputs, execution conditions, producedure and expected results for testing a
  program's behaviour.}\\
\hline
\newterm{Typeform} & Website that is a software as a service that specializes
  in creating and building online surveys.\\
\hline
\newterm{User} & Person who uses or operates a computer program.\\
\hline
\newterm{User Interface} & Where interactions between machines and humans
  occur.\\
\bottomrule
\end{tabularx}
    
\end{table} 

\newpage
\FloatBarrier
\subsection{Overview of Document}
The test plan describes how the team is going to test the software and what
automated testing tools will be used. All \textcolor{red}{
\term[FR]{functional}} and most \textcolor{red}{\term[NFR]{non-functional
requirements}} included in the \textcolor{red}{\term[SRS]{SRS}} will be
tested. Proof of concept tests, as well as comparisons to the existing
\textcolor{red}{\term[PAGAN]{PAGAN}} implementation, unit testing, and
usability survey questions are also described.

\section{Plan}
    
\subsection{Software Description}
The Random Flag Generator \textcolor{red}{(\term[RFG]{RFG})} is a
user-friendly software that allows users to uniquely create a personalized
randomly generated flag based on the entered input string. It is built using
Python and is inspired by \textcolor{red}{\term[PAGAN]{PAGAN}}, an open source
project that randomly generates an avatar from an input string. The user will
be able to select different hash functions and specify different features of
the flag, such as changing the flag image resolution and including certain
symbols, colour\textcolor{red}{s}, and stripes. It will also have a flag
\textcolor{red}{\term[Gallery]{gallery}} that lets users look through
previously created flags.

\subsection{Test Team}
All members of \textcolor{red}{Group \sout{group}} 2 are responsible for
creating, executing, and documenting tests described in this document, as well
as any new tests that are created later. The testing team is comprised of:
\begin{itemize}
    \item Akram Hannoufa
    \item Ganghoon (James) Park
    \item Nathaniel Hu
  \end{itemize}  
Upon completing the final stage of development, TAs, students, and other
volunteers would be asked to provide feedback through the usability survey
questions.

\subsection{Automated Testing Approach}
To review and validate the software product, automated testing for each output
will be conducted. It will be performed automatically by running the
\textcolor{red}{\term[Test Case]{test case}} set. The hash functions along
with the flag generator functions will be tested automatically to ensure that
the input string gives the expected output string and that the result is
repeatable.

\subsection{Testing Tools}
The main software testing framework that will be used is \textcolor{red}{
\term[Pytest]{Pytest}}, which will allow testers to write \textcolor{red}{
\term[Test Case]{test cases}} in \textcolor{red}{\term[Python]{Python}} and
report the test results. It is a simple, efficient and effective way to handle
increasingly complex testing requirements by creating simple unit tests.
\textcolor{red}{\term[Typeform]{Typeform}} will be used to obtain feedback
from different students, TAs, and other volunteers.

\subsection{Testing Schedule}
Please refer to the project Gantt chart: ../ProjectSchedule/3XA3Ganttchart.pdf

\section{System Test Description}
	
\subsection{Tests for Functional Requirements}

\subsubsection{User Interface Testing}
		
\paragraph{User Interface Test}

\begin{enumerate}

\item{FR-01\\}

Type: Functional, Dynamic, Manual

Initial State: UI is uninitialized

Input: command to initialize and run the UI

Output: UI is initialized

How test will be performed: a command will be entered at the command line (by
the \textcolor{red}{tester \sout{user}}). The Random Flag Generator main menu
UI should open and allow the \textcolor{red}{tester \sout{user}} to enter text
into the input string field. The main menu should also present buttons for the
instructions, flag gallery and settings menus.

\item{FR-02\\}

Type: Functional, Dynamic, Manual

Initial State: UI is initialized and running

Input: mouse clicks on the instructions, flag gallery or settings menu
buttons, or the appropriate keyboard strokes assigned to open each menu

Output: the instructions, flag gallery or settings menu will be pulled up and
can be viewed in its entirety

How test will be performed: The \textcolor{red}{tester \sout{user}} will have
the Random Flag Generator main menu UI already running, and then either click
on the buttons or press the appropriate keyboard strokes to pull up each menu.

\item{FR-03\\}

Type: Functional, Dynamic, Manual

Initial State: UI is initialized and running

Input: keyboard strokes to (re)enter text into the input string field

Output: text appears in the input string field

How test will be performed: The \textcolor{red}{tester \sout{user}} will have
the Random Flag Generator main menu UI already running, and then (re)enter
text into the input string field.

\item{FR-04-08\\}

Type: Functional, Dynamic, Manual

Initial State: UI is initialized and running

Input: (text in input string field and) mouse click on generate flag button

Output: flag will be generated using the input string and saved on the local
machine (in \/generated\_flags directory)

How test will be performed: The \textcolor{red}{tester \sout{user}} will have
the Random Flag Generator main menu UI already running, and then click the
generate flag button to start flag generation.

\item{FR-05-09\\}

Type: Functional, Dynamic, Manual

Initial State: UI is initialized and running, a flag has already finished
generating

Input: (text in the input string field and) mouse click on the display flag
button.

Output: generated flag with be displayed to the \textcolor{red}{tester
\sout{user}} through the UI

How test will be performed: The \textcolor{red}{tester \sout{user}} will have
the The Flag Generator main menu UI already running and already have a flag
generated, and then click the display flag button to display the flag in the
UI.

\item{FR-06-10\\}

Type: Functional, Dynamic, Manual

Initial State: UI is initialized and running, a flag has already finished
generating

Input: (text in the input string field,) saved changes to the settings and
mouse click on the display flag button

Output: flag will be regenerated using the input string and changed settings,
and saved on the local machine (in \/generated\_flags directory)

How test will be performed: The \textcolor{red}{tester \sout{user}} will have
the Random Flag Generator main menu UI already running and already have a flag
generated (and displayed). The \textcolor{red}{tester \sout{user}} then makes
and saves some changes to the settings, and reruns the flag generation with the
new settings in effect.

\item{FR-07-11\\}

Type: Functional, Dynamic, Manual

Initial State: UI is initialized and running

Input: (text in the input string field and) mouse click on the generate flag button.

Output: flag will be generated using the input string and saved on the local
machine (in \/generated\_flags directory) using the input string as its
(default) name

How test will be performed: The \textcolor{red}{tester \sout{user}} will have
the Random Flag Generator main menu UI already running, and then click the
generate flag button to start flag generation. The \textcolor{red}{tester
\sout{user}} will then open up the flag gallery menu, and the new generated
flag should be displayed there.

\item{FR-12-13\\}

Type: Functional, Dynamic, Manual

Initial State: UI is initialized and running

Input: mouse click on the instructions button, or the appropriate keyboard
stroke assigned to open the instructions menu

Output: the instructions menu will be pulled up and can be viewed in its
entirety, and a button will be displayed that the \textcolor{red}{tester
\sout{user}} can click on to close the instructions menu and return to the
main menu

How test will be performed: The \textcolor{red}{tester \sout{user}} will have
the Random Flag Generator main menu UI already running, and then either click
on the instructions button or press the appropriate keyboard stroke to pull up
the instructions menu. After viewing the instructions menu, the
\textcolor{red}{tester \sout{user}} will then click on the return to main menu
button to close the instructions window and return to the main menu.

\item{FR-14\\}

Type: Functional, Dynamic, Manual

Initial State: UI is initialized and running

Input: mouse click on the settings button, or the appropriate keyboard stroke
assigned to open the settings menu, and clicks/dragging to change settings

Output: the settings menu will be pulled up where settings can be changed

How test will be performed: The \textcolor{red}{tester \sout{user}} will have
the Random Flag Generator main menu UI already running, and then either click
on the settings button or press the appropriate keyboard stroke to pull up the
settings menu and change settings.

\item{FR-15\\}

Type: Functional, Dynamic, Manual

Initial State: UI is initialized and running

Input: mouse click on the settings button, or the appropriate keyboard stroke
assigned to open the settings menu

Output: the settings menu will be pulled up where the flag generator version
will be displayed

How test will be performed: The \textcolor{red}{tester \sout{user}} will have
the Random Flag Generator main menu UI already running, and then either click
on the settings button or press the appropriate keyboard stroke to pull up the
settings menu to confirm that the flag generator version number is displayed
properly.

\item{FR-16\\}

Type: Functional, Dynamic, Manual

Initial State: UI is initialized and running

Input: mouse click on the settings button, or the appropriate keyboard stroke
assigned to open the settings menu

Output: the settings menu will be pulled up where a return to main menu button
will be displayed that the \textcolor{red}{tester \sout{user}} can click on to
close the settings menu and return to the main menu

How test will be performed: The \textcolor{red}{tester \sout{user}} will have
the Random Flag Generator main menu UI already running, and then either click
on the settings button or press the appropriate keyboard stroke to pull up the
settings menu. After viewing the settings menu, the \textcolor{red}{tester
\sout{user}} will then click on the return to main menu button to close the
instructions window and return to the main menu.

\item{FR-17\\}

Type: Functional, Dynamic, Manual

Initial State: UI is initialized and running

Input: mouse click on the flag gallery button, or the appropriate keyboard
stroke assigned to open the flag gallery menu

Output: the flag gallery menu will be pulled up where a list of all flags and
the input strings used to generate them will be displayed

How test will be performed: The \textcolor{red}{tester \sout{user}} will have
the Random Flag Generator main menu UI already running, and then either click
on the flag gallery button or press the appropriate keyboard stroke to pull up
the flag gallery menu for viewing.

\item{\textcolor{red}{\sout{FR-18}}\\}

\textcolor{red}{\sout{Type: Functional, Dynamic, Manual}}

\textcolor{red}{\sout{Initial State: UI is initialized and running}}

\textcolor{red}{\sout{Input: mouse click on the flag gallery button, or the
appropriate keyboard stroke assigned to open the flag gallery menu}}

\textcolor{red}{\sout{Output: the flag gallery menu will be pulled up where an
input string field will allow the user to search for flags by input string}}

\textcolor{red}{\sout{How test will be performed: The user will have the
Random Flag Generator main menu UI already running, and then either click on
the flag gallery button or press the appropriate keyboard stroke to pull up
the flag gallery menu. The user will then input text in the input string field
to search for flags by input string.}}

\item{FR-19\\}

Type: Functional, Dynamic, Manual

Initial State: UI is initialized and running

Input: mouse click on the flag gallery button, or the appropriate keyboard
stroke assigned to open the flag gallery menu

Output: the flag gallery menu will be pulled up where a button will be
displayed that the \textcolor{red}{tester \sout{user}} can click on to close
the flag gallery menu and return to the main menu

How test will be performed: The \textcolor{red}{tester \sout{user}} will have
the Random Flag Generator main menu UI already running, and then either click
on the flag gallery button or press the appropriate keyboard stroke to pull up
the flag gallery menu. After viewing the flag gallery menu, the
\textcolor{red}{tester \sout{user}} will then click on the return to main menu
button to close the flag gallery window and return to the main menu.

\end{enumerate}

\subsubsection{Output Testing}
		
\paragraph{Output Test}

\begin{enumerate}

\item{test\_get\_hash\_algo\\}

Type: Functional, Dynamic, Automated

Initial State:

Input: input string of hashing algorithm name

Output: hashlib hashing algorithm

How test will be performed: automatically by running the test program
containing the test set

\item{test\_get\_hash\_hex\\}

Type: Functional, Dynamic, Automated

Initial State:

Input: input string to be used to generate flag, hashlib hashing algorithm

Output: hash digest of input string in hexadecimal form

How test will be performed: automatically by running the test program
containing the test set

\item{test\_hash\_generator\\}

Type: Functional, Dynamic, Automated

Initial State:

Input: input string to be used to generate flag

Output: hash digest of input string in hexadecimal form

How test will be performed: automatically by running the test program
containing the test set

\item{test\_pad\_hashcode\\}

Type: Functional, Dynamic, Automated

Initial State:

Input: output hash string to be used to generate flag

Output: output hash string padded to minimum hash length

How test will be performed: automatically by running the test program
containing the test set

\item{\textcolor{red}{test\_choose\_from\_list\\}}

\textcolor{red}{Type: Functional, Dynamic, Automated}

\textcolor{red}{Initial State:}

\textcolor{red}{Input: list of elements, float value to determine an index to
be used to select an element from the list}

\textcolor{red}{Output: element from the list with index larger than and
closest to input float value}

\textcolor{red}{How test will be performed: automatically by running the test
program containing the test set}

\item{\textcolor{red}{test\_map\_decision\\}}

\textcolor{red}{Type: Functional, Dynamic, Automated}

\textcolor{red}{Initial State:}

\textcolor{red}{Input: three numerical values representing the maximum
possible option, the number of possible decisions and the digit to map within
the possible options}

\textcolor{red}{Output: float index value to be used to decide which element
to select (presumably from a list)}

\textcolor{red}{How test will be performed: automatically by running the test
program containing the test set}

\item{\textcolor{red}{test\_split\_sequence\\}}

\textcolor{red}{Type: Functional, Dynamic, Automated}

\textcolor{red}{Initial State:}

\textcolor{red}{Input: input string, integer value specifying the length of
the substrings to be split from the input string}

\textcolor{red}{Output: list of substrings from the input string with as many
substrings of the specified length as possible}

\textcolor{red}{How test will be performed: automatically by running the test
program containing the test set}

\item{test\_grind\_hash\_for\_colors\\}

Type: Functional, Dynamic, Automated

Initial State:

Input: output hash string to be used to generate flag

Output: list of five colors' RGB values (R, G, B)

How test will be performed: automatically by running the test program
containing the test set

\item{test\_grind\_hash\_for\_\textcolor{red}{base\_}stripe\_style\\}

Type: Functional, Dynamic, Automated

Initial State:

Input: output hash string to be used to generate flag

Output: output string of \textcolor{red}{base} stripe style to be used to
generate flag

How test will be performed: automatically by running the test program
containing the test set

\item{\textcolor{red}{test\_grind\_hash\_for\_overlay\_stripe\_style\\}}

\textcolor{red}{Type: Functional, Dynamic, Automated}

\textcolor{red}{Initial State:}

\textcolor{red}{Input: output hash string to be used to generate flag}

\textcolor{red}{Output: output string of overlay stripe style to be used to
generate flag}

\textcolor{red}{How test will be performed: automatically by running the test
program containing the test set}

\item{test\_grind\_hash\_for\_stripe\_number\\}

Type: Functional, Dynamic, Automated

Initial State:

Input: output hash string to be used to generate flag

Output: output string of stripe number to be used to generate flag

How test will be performed: automatically by running the test program
containing the test set

\item{test\_grind\_hash\_for\_symbol\_locations\\}

Type: Functional, Dynamic, Automated

Initial State:

Input: output hash string to be used to generate flag

Output: output string of symbol location to be used to generate flag

How test will be performed: automatically by running the test program
containing the test set

\item{test\_grind\_hash\_for\_symbol\_number\\}

Type: Functional, Dynamic, Automated

Initial State:

Input: output hash string to be used to generate flag

Output: output string of symbol number to be used to generate flag

How test will be performed: automatically by running the test program
containing the test set

\item{test\_grind\_hash\_for\_symbol\_types\\}

Type: Functional, Dynamic, Automated

Initial State:

Input: output hash string to be used to generate flag

Output: output string of symbol type to be used to generate flag

How test will be performed: automatically by running the test program
containing the test set

\item{test\_hex2rgb\\}

Type: Functional, Dynamic, Automated

Initial State:

Input: output hexidecimal color code string to be used to generate flag

Output: output tuple of RGB values for color derived from heximdecimal color
code string to be used to generate flag

How test will be performed: automatically by running the test program
containing the test set

\item{\textcolor{red}{test\_diff\\}}

\textcolor{red}{Type: Functional, Dynamic, Automated}

\textcolor{red}{Initial State:}

\textcolor{red}{Input: two float values}

\textcolor{red}{Output: absolute value of the difference between the two given
float values}

\textcolor{red}{How test will be performed: automatically by running the test
program containing the test set}

\item{test\_generate\_flag\\}

Type: Functional, Dynamic, \textcolor{red}{Automated \sout{Manual}}

Initial State:

Input: input string, hashing algorithm name string \textcolor{red}{and
dictionary of settings (i.e. chosen colours, elements)} to be used to generate
flag

Output: generated flag image file

How test will be performed: \textcolor{red}{automatically by running the test
program containing the test set \sout{manually by running the test program
containing the test set, and then doing a manual visual comparison of the
generated flag with its expected generated flag output}}

\item{test\_generate\_flag\_data\\}

Type: Functional, Dynamic, Automated

Initial State:

Input: input string, hashing algorithm name string to be used to generate flag

Output: tuple consisting of list of 5 tuples of RGB values, tuple of stripe
style, number and tuple of symbol location, number and type

How test will be performed: automatically by running the test program
containing the test set

\item{\textcolor{red}{test\_parse\_jka\_file\\}}

\textcolor{red}{Type: Functional, Dynamic, Automated}

\textcolor{red}{Initial State:}

\textcolor{red}{Input: input string of flag asset (.jka) file name to be
parsed for pixel data}

\textcolor{red}{Output: list consisting of tuples of (x, y) coordinates of the
position of all filled pixels that compose the selected flag asset}

\textcolor{red}{How test will be performed: automatically by running the test
program containing the test set}

\end{enumerate}

\subsection{Tests for Nonfunctional Requirements}

\subsubsection{Look and Feel Testing}
		
\paragraph{Appearance Test}

\begin{enumerate}

\item{LF-01\\}

Type: Manual, Dynamic

Initial State: UI is initialized and running

Input/Condition: 

Output/Result: verification of UI simplicity, intuitivity, and clutterlessness

How test will be performed: The Random Flag Generator UI will be analyzed for
how simple, intuitive and non-cluttered it is for users by several
\textcolor{red}{testers \sout{users}}. The \textcolor{red}{testers
\sout{users}} will provide their answer to question 1 of the Usability Survey.
Their responses will be assessed to determine if the UI is simple, intuitive
and not cluttered.

\end{enumerate}

\paragraph{Style Test}

\begin{enumerate}

\item{LF-02\\}

Type: Manual, Functional, Dynamic

Initial State: UI is initialized and running

Input/Condition: input string, hashing algorithm name string to generate a
flag

Output/Result: flag generated using input string and hashing algorithm

How test will be performed: The \textcolor{red}{testers \sout{users}} will
enter input strings and hashing algorithm string names to generate flags. The
\textcolor{red}{testers \sout{users}} will then review the generated flags for
image quality and aesthetics, and provide their answer to question 2 of the
Usability Survey. Their responses will be assessed to gauge how aesthetically
pleasing the generated flags look to the users.

\item{LF-03\\}

Type: Manual, Functional, Dynamic

Initial State: UI is initialized and running

Input/Condition: input string, hashing algorithm name string to generate a
flag

Output/Result: flag generated using input string and hashing algorithm

How test will be performed: The \textcolor{red}{tester \sout{user}} will
enter an input string and hashing algorithm string name to generate a flag.
\textcolor{red}{The \sout{Then} tester \sout{user}} will then review the
generated flag to ensure the colours used are within certain colour ranges.

\item{LF-04\\}

Type: Manual, Functional, Dynamic

Initial State: UI is initialized and running

Input/Condition: input string, hashing algorithm name string to generate a
flag

Output/Result: flag generated using input string and hashing algorithm

How test will be performed: The \textcolor{red}{tester \sout{user}} will enter
an input string and hashing algorithm string name to generate a flag.
\textcolor{red}{The \sout{Then} tester \sout{user}} will then review the
generated flag to ensure all flag components are visible in the generated
image.

\end{enumerate}

\subsubsection{Usability and Humanity Testing}

\paragraph{Ease of Use Test}

\begin{enumerate}

\item{UH-01\\}

Type: Manual, Dynamic

Initial State: UI is initialized and running

Input/Condition: 

Output/Result: verification of the logical flow of the UI components

How test will be performed: The Random Flag Generator UI will be analyzed for
the logical flow of the UI components by several \textcolor{red}{testers
\sout{users}}. The \textcolor{red}{testers \sout{users}} will then provide
their answer to question 3 of the Usability Survey. Their responses will be
assessed to determine if the UI components flow follow a logical flow.

\item{UH-02\\}

Type: Manual, Functional, Dynamic

Initial State: UI is initialized and running

Input/Condition: 

Output/Result: verification that all tasks can be done within one interface

How test will be performed: Various tasks will be performed (e.g. reading the
instructions, changing the settings) \textcolor{red}{by testers} to ensure all
tasks can be done within one interface, and that no tasks need interactions
with more than one interface to be completed.

\item{UH-03\\}

Type: Manual, Dynamic

Initial State: UI is initialized and running

Input/Condition: 

Output/Result: verification that the Random Flag Generator is easy to use for
people aged \textcolor{red}{$\hyperlink{min_age}{MIN\_AGE}$ \sout{7}} or older

How test will be performed: Various tasks will be performed (e.g. reading the
instructions, changing the settings) \textcolor{red}{by testers} to ensure all
tasks can be easily understood and completed by people aged \textcolor{red}{
$\hyperlink{min_age}{MIN\_AGE}$ \sout{7}} or older.

\item{\textcolor{red}{\sout{UH-04}}\\}

\textcolor{red}{\sout{Type: Manual, Dynamic}}

\textcolor{red}{\sout{Initial State: UI is initialized and running}}

\textcolor{red}{\sout{Input/Condition:}}

\textcolor{red}{\sout{Output/Result: verification that all keyboard inputs to
perform actions are intuitive for users}}

\textcolor{red}{\sout{How test will be performed: Various tasks will be
performed (e.g. reading the instructions, changing the settings) to ensure all
tasks can be easily understood and completed using keyboard inputs that are
intuitive and easy to remember and use.}}

\end{enumerate}

\paragraph{Personalization Test}

\begin{enumerate}

\item{UH-05\\}

Type: Manual, Functional, Dynamic

Initial State: UI is initialized and running

Input/Condition: 

Output/Result: verification of the modifiability of output specifications for
flag generation (e.g. type of hashing, image file, etc.)

How test will be performed: The Random Flag Generator UI will be analyzed by
several \textcolor{red}{testers \sout{users}} to ensure the output
specifications for flag generation can (easily) be modified. The
\textcolor{red}{testers \sout{users}} will then provide their answer to
question 5 of the Usability Survey. Their responses will be assessed to
determine how much more or less ability the users should have to set certain
parts of the generated flag.

\item{\textcolor{red}{\sout{UH-06}}\\}

\textcolor{red}{\sout{Type: Manual, Functional, Dynamic}}

\textcolor{red}{\sout{Initial State: UI is initialized and running}}

\textcolor{red}{\sout{Input/Condition:}}

\textcolor{red}{\sout{Output/Result: verification of the modifiability of the
flag gallery}}

\textcolor{red}{\sout{How test will be performed: The Random Flag Generator UI
will be analyzed by several users to ensure the flag gallery can (easily) be
modified (i.e. add, delete flags). The users will then provide their answer to
question 6 of the Usability Survey. Their responses will be assessed to
determine the value of and ease of use of the flag gallery.}}

\end{enumerate}

\paragraph{Learning Test}

\begin{enumerate}

\item{UH-07\\}

Type: Manual, Dynamic

Initial State: UI is initialized and running

Input/Condition: 

Output/Result: verification of the learnability of the Random Flag Generator
program

How test will be performed: The Random Flag Generator UI will be analyzed by
several users to ensure they can learn to use the program with no prior
experience. The users with no prior experience will also attempt to use the
program \textcolor{red}{\sout{(and 85\% of users should be able to with
ease)}}. The users will then provide their answer to question 8 of the
Usability Survey. Their responses will be assessed to determine what parts of
the GUI were unclear or could be improved.

\item{UH-08\\}

Type: Manual, Dynamic

Initial State: UI is initialized and running

Input/Condition: 

Output/Result: verification of the accessibility of the instructions menu

How test will be performed: The Random Flag Generator UI will be analyzed by
several users to ensure they can (easily) access the instruction menu for
help. The users will also attempt to use the program (and \textcolor{red}{
\sout{85\% of users}} should be able to pull up the instructions menu within
\textcolor{red}{$\hyperlink{max_find_time}{MAX\_FIND\_TIME}$ \sout{5}
seconds)}. The users will then provide their answer to question 4 of the
Usability Survey. Their responses will be assessed to determine if the
instructions/help menu button was helpful to users or not and how it could be
improved.

\item{UH-09\\}

Type: Manual, Dynamic

Initial State: UI is initialized and running

Input/Condition: 

Output/Result: verification of the intuitivity/informativeness of the main UI

How test will be performed: The Random Flag Generator UI will be analyzed by
several \textcolor{red}{testers \sout{users}} to ensure the user can (easily)
use the program by following the main UI instructions. A sample group of users
will also attempt to use the program (and \textcolor{red}{\sout{85\% of
users}} should be able to do so just using the main UI instructions).

\end{enumerate}

\paragraph{Understandability Test}

\begin{enumerate}

\item{UH-10\\}

Type: Manual, Dynamic

Initial State: UI is initialized and running

Input/Condition: 

Output/Result: verification of consistent language use throughout the program

How test will be performed: The Random Flag Generator UI will be analyzed
\textcolor{red}{by testers} to ensure that consistent language is used
throughout the program.

\item{UH-11\\}

Type: Manual, Dynamic

Initial State: UI is initialized and running

Input/Condition: 

Output/Result: verification of simplified terminology being used wherever
possible

How test will be performed: The Random Flag Generator UI will be analyzed
\textcolor{red}{by testers} to ensure that simplified terminology is used
wherever possible in the program.

\end{enumerate}

\paragraph{Accessibility Test}

\begin{enumerate}

\item{UH-12\\}

Type: Manual, Dynamic

Initial State: UI is initialized and running

Input/Condition:

Output/Result: verification of easy-to-read fonts and font sizes being used
throughout the program

How test will be performed: The Random Flag Generator UI will be analyzed
\textcolor{red}{by testers} and compared with the web accessibility standards
to ensure that easy-to-read fonts and font sizes are used throughout the
program.

\item{\textcolor{red}{\sout{UH-13}}\\}

\textcolor{red}{\sout{Type: Manual, Dynamic}}

\textcolor{red}{\sout{Initial State: UI is initialized and running}}

\textcolor{red}{\sout{Input/Condition:}}

\textcolor{red}{\sout{Output/Result: verification of a colour-blind friendly
Random Flag Generator UI}}

\textcolor{red}{\sout{How test will be performed: The Random Flag Generator UI
will be analyzed to ensure that a colour-blind friendly UI is provided.}}

\end{enumerate}

\subsubsection{Performance Testing}

\paragraph{Speed Test}

\begin{enumerate}

\item{PE-01-02\\}

Type: Manual, Functional, Dynamic

Initial State: UI is initialized and running

Input/Condition: input string and hashing algorithm name string to generate
flag, mouse click

Output/Result: output generated flag image

How test will be performed: The Random Flag Generator program will be given an
input string and hashing algorithm name string, and the time it takes to
generate/download a flag image shall be timed. The generation and download
time shall not exceed \textcolor{red}{
$\hyperlink{max_load_time}{MAX\_LOAD\_TIME}$ \sout{5}} seconds.

\item{PE-03\\}

Type: Manual, Functional, Dynamic

Initial State: UI is initialized and running

Input/Condition: mouse click

Output/Result: user's flag gallery should be loaded in the UI

How test will be performed: The Random Flag Generator UI program will open the
user's flag gallery and the duration shall be timed. The loading time shall
not exceed \textcolor{red}{$\hyperlink{max_load_time}{MAX\_LOAD\_TIME}$
\sout{5}} seconds.

\end{enumerate}

\paragraph{Precision or Accuracy Test}

\begin{enumerate}

\item{PE-04\\}

Type: Manual, Functional, Dynamic

Initial State: UI is initialized and running

Input/Condition: input string and hashing algorithm name string to generate
flag

Output/Result: output hexadecimal hash digest (per the selected hashing
algorithm)

How test will be performed: The Random Flag Generator program will be given an
input string and hashing algorithm name string, and the output hexadecimal
hash digest (per the selected hashing algorithm) will be compared with the
hash digest generated with the input string using an external hashing program.

\item{PE-05\\}

Type: Manual, Functional, Dynamic

Initial State: UI is initialized and running

Input/Condition: input string and hashing algorithm name string to generate
flag, mouse click

Output/Result: output generated flag image

How test will be performed: The Random Flag Generator program will be given an
input string and hashing algorithm name string, and the generated flag image
will be analyzed to ensure the various components are accurately placed (i.e.
they match the template files).

\item{PE-06\\}

Type: Manual, Functional, Dynamic

Initial State: UI is initialized and running

Input/Condition: input string and hashing algorithm name string to generate
flag, mouse click

Output/Result: output generated flag image

How test will be performed: The Random Flag Generator program will be given an
input string and hashing algorithm name string, and the generated flag image
will be analyzed to ensure the various colours used match the hexadecimal RGB
colour values derived from the hash digest. The RGB volour values will be run
through an external colour tool, and those colours will be compared with the
generated flag's colours to ensure they match.

\end{enumerate}

\paragraph{Reliability and Availability Test}

\begin{enumerate}

\item{PE-07\\}

Type: Manual, Dynamic

Initial State: GUI is uninitialized

Input/Condition: command to initialize and run the GUI

Output/Result: GUI is initialized and running

How test will be performed: The Random Flag Generator program will be
initialized and run from the command line at several different times during
the day to ensure its availability.

\end{enumerate}

\paragraph{Capacity Test}

\begin{enumerate}

\item{\textcolor{red}{\sout{PE-08}}\\}

\textcolor{red}{\sout{Type: Manual, Functional, Dynamic}}

\textcolor{red}{\sout{Initial State: UI is initialized and running, maximum
number of generated flags currently stored in user's gallery}}

\textcolor{red}{\sout{Input/Condition: input string and hashing algorithm name
string to generate flag, mouse click}}

\textcolor{red}{\sout{Output/Result: program displays an error message telling
the user the maximum number of generated flags has been reached, and to delete
flags in the gallery to make more space for new flags.}}

\textcolor{red}{\sout{How test will be performed: The Random Flag Generator
program will be running and the user will attempt to generate a new flag
despite the gallery being full (i.e. the flag generation should fail and an
error message should pop up).}}

\item{PE-09\\}

Type: Manual, Functional, Dynamic

Initial State: GUI is uninitialized

Input/Condition: command to initialize and run the GUI

Output/Result: GUI is initialized and running

How test will be performed: The Random Flag Generator program will be
initialized and run from the command line, and only one user is initialized
per machine (i.e. all generated flags are stored in the same gallery/directory
on the machine).

\end{enumerate}

\paragraph{Scalability or Extensibility Test}

\begin{enumerate}

\item{PE-10\\}

Type: Manual, Functional, Dynamic

Initial State: GUI is uninitialized

Input/Condition: a new hashing algorithm/function option is added, command to
initialize and run the GUI

Output/Result: program should be able to generate new flags using the new
hashing algorithm/function without any changes needing to be made to the
existing code.

How test will be performed: A new hashing algorithm/function will be added to
the Random Flag Generator program, and then the GUI will be initialized. The
user will attempt to generate a new flag using the new hashing algorithm and
an input string, and the generated flag should be different from all of the
other flags generated using that input string and all other existing hashing
algorithms.

\item{PE-11\\}

Type: Manual, Functional, Dynamic

Initial State: GUI is uninitialized

Input/Condition: new flag components are added (i.e. new .jka flag asset
files), command to initialize and run the GUI

Output/Result: program should be able to generate new flags using the new flag
asset files with only minor to no changes being made to the existing code.

How test will be performed: A new (or several) flag asset(s) will be added to
the Random Flag Generator program, and then the GUI will be initialized. The
users will attempt to generate a new flag using the new flag asset(s) with
specific input strings (determining by trial and error), and the generated
flag should use the new flag asset(s). The users will then provide their
answer to question 7 of the Usability Survey. Their responses will be assessed
to determine what new designs the users would like to see become available.

\end{enumerate}

\subsubsection{Operational and Environmental Testing}

\paragraph{Expected Physical Environment Test}

\begin{enumerate}

\item{PE-13\\}

Type: Manual, Functional, Dynamic

Initial State: GUI is uninitialized

Input/Condition: local machine has Wi-Fi disabled, command to initialize and
run the GUI

Output/Result: GUI is initialized and running

How test will be performed: The Random Flag Generator program will be
initialized and run from the command line on a local machine with the Wi-Fi
disabled. This program should still function as it would even with the Wi-Fi
enabled.

\item{PE-14\\}

Type: Manual, Functional, Dynamic

Initial State: GUI is uninitialized

Input/Condition: local machine can support (running) the Python language,
command to initialize and run the GUI

Output/Result: GUI is initialized and running

How test will be performed: The Random Flag Generator program will be
initialized and run from the command line on several different local machines
that can support (running) the Python language. This program should still
function the same on all the different local machines.

\end{enumerate}

\subsubsection{Interfacing with Adjacent Systems Testing}

\paragraph{Interfacing with Adjacent Systems Test}

\begin{enumerate}

\item{PE-15\\}

Type: Manual, Functional, Dynamic

Initial State: GUI is uninitialized

Input/Condition: command to initialize and run the GUI, user generates some
flag(s)

Output/Result: GUI is initialized and running, new flag(s) generated and saved
to the user's local machine

How test will be performed: The Random Flag Generator program will be
initialized and run from the command line, and the user will attempt to
generate some flag(s). The user will then check that the program has only
created/altered files within its working directory (i.e. generated\_flags
directory), and no where else on the local machine.

\end{enumerate}

\subsection{Traceability Between Test Cases and Requirements}

In terms of traceability, most \textcolor{red}{\term[Test Case]{test cases}}
and requirements have a one to one relationship, with exceptions and special
cases noted. Some \textcolor{red}{\term[FR]{functional requirements}} are
covered using the same tests, because they are essentially the same
requirements, but are a part of different business events. See the following
traceability matrices for more details.

\begin{landscape}

\begin{table}[h!]
\centering
\caption{\textbf{Traceability Matrix for Functional Requirements I}}
\label{tab:trace_matrix_01}
\begin{tabular}{|c|c|c|c|c|c|c|c|c|c|c|c|c|c|c|c|}
\hline
& \multicolumn{14}{c|}{Requirements} \\
\hline
Test Cases & FR1 & FR2 & FR3 & FR4 & FR5 & FR6 & FR7 & FR8 & FR9 & FR10 & FR11 & FR12 & FR13 & FR14 \\
\hline
FR-01 & X & & & & & & & & & & & & & \\
\hline
FR-02 & & X & & & & & & & & & & & & \\
\hline
FR-03 & & & X & & & & & & & & & & & \\
\hline
FR-04-08 & & & & X & & & & X & & & & & & \\
\hline
FR-05-09 & & & & & X & & & & X & & & & & \\
\hline
FR-06-10 & & & & & & X & & & & X & & & & \\
\hline
FR-07-11 & & & & & & & X & & & & X & & & \\
\hline
FR-12-13 & & & & & & & & & & & & X & X & \\
\hline
FR-14 & & & & & & & & & & & & & & X \\
\hline
\end{tabular}
\end{table}

\newpage

\begin{table}[h!]
\centering
\caption{\textbf{Traceability Matrix for Functional Requirements II/Non-Functional Requirements I}}
\label{tab:trace_matrix_02}
\begin{tabular}{|c|c|c|c|c|c|c|c|c|c|c|c|c|c|c|}
\hline
& \multicolumn{14}{c|}{Requirements} \\
\hline
Test Cases & FR15 & FR16 & FR17 & \textcolor{red}{\sout{FR18}} & FR19 & LF1 & LF2 & LF3 & LF4 & UH1 & UH2 & UH3 & \textcolor{red}{\sout{UH4}} & UH5 \\
\hline
FR-15 & X & & & & & & & & & & & & & \\
\hline
FR-16 & & X & & & & & & & & & & & & \\
\hline
FR-17 & & & X & & & & & & & & & & & \\
\hline
\textcolor{red}{\sout{FR-18}} & & & & \textcolor{red}{\sout{X}} & & & & & & & & & & \\
\hline
FR-19 & & & & & X & & & & & & & & & \\
\hline
LF-01 & & & & & & X & & & & & & & & \\
\hline
LF-02 & & & & & & & X & & & & & & & \\
\hline
LF-03 & & & & & & & & X & & & & & & \\
\hline
LF-04 & & & & & & & & & X & & & & & \\
\hline
UH-01 & & & & & & & & & & X & & & & \\
\hline
UH-02 & & & & & & & & & & & X & & & \\
\hline
UH-03 & & & & & & & & & & & & X & & \\
\hline
\textcolor{red}{\sout{UH-04}} & & & & & & & & & & & & & \textcolor{red}{\sout{X}} & \\
\hline
UH-05 & & & & & & & & & & & & & & X \\
\hline
\end{tabular}
\end{table}

\newpage

\begin{table}[h!]
\centering
\caption{\textbf{Traceability Matrix for Non-Functional Requirements II}}
\label{tab:trace_matrix_03}
\begin{tabular}{|c|c|c|c|c|c|c|c|c|c|c|c|c|c|c|}
\hline
& \multicolumn{14}{c|}{Requirements} \\
\hline
Test Cases & \textcolor{red}{\sout{UH6}} & UH7 & UH8 & UH9 & UH10 & UH11 & UH12 & \textcolor{red}{\sout{UH13}} & PE1 & PE2 & PE3 & PE4 & PE5 & PE6 \\
\hline
\textcolor{red}{\sout{UH-06}} & \textcolor{red}{\sout{X}} & & & & & & & & & & & & & \\
\hline
UH-07 & & X & & & & & & & & & & & & \\
\hline
UH-08 & & & X & & & & & & & & & & & \\
\hline
UH-09 & & & & X & & & & & & & & & & \\
\hline
UH-10 & & & & & X & & & & & & & & & \\
\hline
UH-11 & & & & & & X & & & & & & & & \\
\hline
UH-12 & & & & & & & X & & & & & & & \\
\hline
\textcolor{red}{\sout{UH-13}} & & & & & & & & \textcolor{red}{\sout{X}} & & & & & & \\
\hline
PE-01-02 & & & & & & & & & X & X & & & & \\
\hline
PE-03 & & & & & & & & & & & X & & & \\
\hline
PE-04 & & & & & & & & & & & & X & & \\
\hline
PE-05 & & & & & & & & & & & & & X & \\
\hline
PE-06 & & & & & & & & & & & & & & X \\
\hline
\end{tabular}
\end{table}

\newpage

\begin{table}[h!]
\centering
\caption{\textbf{Traceability Matrix for Non-Functional Requirements III}}
\label{tab:trace_matrix_03}
\begin{tabular}{|c|c|c|c|c|c|c|c|c|c|c|c|c|c|c|}
\hline
& \multicolumn{8}{c|}{Requirements} \\
\hline
Test Cases & PE7 & \textcolor{red}{\sout{PE8}} & PE9 & PE10 & PE11 & PE13 & PE14 & PE15 \\
\hline
PE-07 & X & & & & & & & \\
\hline
\textcolor{red}{\sout{PE-08}} & & \textcolor{red}{\sout{X}} & & & & & & \\
\hline
PE-09 & & & X & & & & & \\
\hline
PE-10 & & & & X & & & & \\
\hline
PE-11 & & & & & X & & & \\
\hline
PE-13 & & & & & & X & & \\
\hline
PE-14 & & & & & & & X & \\
\hline
PE-15 & & & & & & & & X \\
\hline
\end{tabular}
\end{table}

\end{landscape}

\section{Tests for Proof of Concept}

\subsection{Hash Generator}
		
\paragraph{test\_get\_hash\_algo}

\begin{enumerate}

\item{test-01\\}

Type: Functional, Dynamic, Automated

Initial State:

Input: 'sha256'

Output: hashlib.sha256

How test will be performed: automatically by running the test program
containing the test set

\item{test-02\\}

Type: Functional, Dynamic, Automated

Initial State:

Input: 'md5'

Output: hashlib.md5

How test will be performed: automatically by running the test program
containing the test set

\end{enumerate}

\paragraph{test\_get\_hash\_hex}

\begin{enumerate}

\item{test-03\\}

Type: Functional, Dynamic, Automated

Initial State:

Input: ‘test’, hashlib.sha256

Output: '9f86d081884c7d659a2feaa0c55ad015a3bf4f1b2b0b822cd15d6c15b0f00a08'

How test will be performed: automatically by running the test program
containing the test set

\end{enumerate}

\paragraph{test\_hash\_generator}

\begin{enumerate}

\item{test-04\\}

Type: Functional, Dynamic, Automated

Initial State:

Input: ‘sample'

Output: 'af2bdbe1aa9b6ec1e2ade1d694f41fc71a831d0268e9891562113d8a62add1bf'

How test will be performed: automatically by running the test program
containing the test set

\end{enumerate}

\subsection{Hash To Flag}

\paragraph{test\_pad\_hashcode}

\begin{enumerate}

\item{test-05\\}

Type: Functional, Dynamic, Automated

Initial State:

Input: ‘575758’

Output: '575758575758575758575758575758'

How test will be performed: automatically by running the test program
containing the test set

\end{enumerate}

\paragraph{test\_grind\_hash\_for\_colors}

\begin{enumerate}

\item{test-06\\}

Type: Functional, Dynamic, Automated

Initial State:

Input: ‘0396233d5b28eded8e34c1bf9dc80fae34756743594b9e5ae67f4f7d124d2e3d’

Output: [(3, 150, 35), (61, 91, 40), (237, 237, 142), (52, 193, 191), (157,
200, 15)]

How test will be performed: automatically by running the test program
containing the test set

\end{enumerate}

\paragraph{test\_grind\_hash\_for\_stripe\_style}

\begin{enumerate}

\item{test-07\\}

Type: Functional, Dynamic, Automated

Initial State:

Input: '5555556c48be1c0b87a7d575c73f6e42fnfhasdf341123abadbdbd'

Output: 'VERTICAL'

How test will be performed: automatically by running the test program
containing the test set

\end{enumerate}

\paragraph{test\_grind\_hash\_for\_stripe\_number}

\begin{enumerate}

\item{test-08\\}

Type: Functional, Dynamic, Automated

Initial State:

Input: '5555556c48be1c0b87a7d575c73f6e42fnfhasdf341123abadbdbd'

Output: 'THREE'

How test will be performed: automatically by running the test program
containing the test set

\end{enumerate}

\paragraph{test\_grind\_hash\_for\_symbol\_locations}

\begin{enumerate}

\item{test-09\\}

Type: Functional, Dynamic, Automated

Initial State:

Input: '5555556c48be1c0b87a7d575c73f6e42fnfhasdf341123abadbdbd'

Output: 'TOP\_LEFT'

How test will be performed: automatically by running the test program
containing the test set

\end{enumerate}

\paragraph{test\_grind\_hash\_for\_symbol\_number}

\begin{enumerate}

\item{test-10\\}

Type: Functional, Dynamic, Automated

Initial State:

Input: '5555556c48be1c0b87a7d575c73f6e42fnfhasdf341123abadbdbd'

Output: 'ONE'

How test will be performed: automatically by running the test program
containing the test set

\end{enumerate}

\paragraph{test\_grind\_hash\_for\_symbol\_types}

\begin{enumerate}

\item{test-11\\}

Type: Functional, Dynamic, Automated

Initial State:

Input: ‘000010000100000010000000999999’

Output: 'CROSS'

How test will be performed: automatically by running the test program
containing the test set

\end{enumerate}

\paragraph{test\_hex2rgb}

\begin{enumerate}

\item{test-12\\}

Type: Functional, Dynamic, Automated

Initial State:

Input: '\#ffffff'

Output: (255, 255, 255)

How test will be performed: automatically by running the test program
containing the test set

\end{enumerate}

\subsection{Flag Generator}

\paragraph{test\_generate\_flag}

\begin{enumerate}

\item{test-13\\}

Type: Functional, Dynamic, Manual

Initial State:

Input: 'Sword is pointy', 'sha256'

Output: flag\_Sword is pointy.png

How test will be performed: manually by running the test program containing
the test set, and then doing a manual visual comparison of the generated flag
with its expected generated flag output

\end{enumerate}

\paragraph{test\_generate\_flag\_data}

\begin{enumerate}

\item{test-14\\}

Type: Functional, Dynamic, Automated

Initial State:

Input: 'test', 'sha256'

Output: ([(159, 134, 208), (129, 136, 76), (125, 101, 154), (47, 234, 160),
(197, 90, 208)], ('VERTICAL', 'THREE'), ('CENTER', 'ZERO', 'CROSS'))

How test will be performed: automatically by running the test program
containing the test set

\end{enumerate}

\subsection{GUI}

\paragraph{test\_GUI}

\begin{enumerate}

\item{test-15\\}

Type: Functional, Dynamic, Manual

Initial State: GUI is uninitialized

Input: command to initialize and run the GUI

Output: GUI is initialized

How test will be performed: a command will be entered at the command line (by
the user). The Random Flag Generator main menu UI should open and allow the
user to enter text into the input string field. The main menu should also
present buttons for generating and displaying flags, which the user will then
click on to test that they are functioning as prescribed.

\end{enumerate}

\section{Comparison to Existing Implementation}

The Random Flag Generator \textcolor{red}{(\term[FRG]{RFG})} program currently
implements the majority of the core functionality found in the original
\textcolor{red}{\term[PAGAN]{PAGAN}} program. Mainly, the Random
Flag Generator program currently implements hash generation using hashlib,
mapping the \textcolor{red}{generated} hash \textcolor{red}{digest} to various
flag design decisions (array indices), and generating the flag image. In
addition, a minimal \textcolor{red}{\term[GUI]{GUI}} has been implemented to
handle user input and display flag output. The next steps are expanding
\textcolor{red}{\term[Test Case]{test cases}} and test coverage, implementing
more flag design options (including more resolution choices), and enhancing
the \textcolor{red}{\term[GUI]{GUI}}, which includes the addition of a flag
gallery.

\section{Unit Testing Plan}
		
\subsection{Unit testing of internal functions}

The \textcolor{red}{\term[Pytest]{Pytest}} library will be used to carry out
all unit testing for internal functions used in the Random Flag Generator
\textcolor{red}{(\term[RFG]{RFG})} program. 5 \textcolor{red}{
\term[Python]{Python}} files will be made that will contain the
\textcolor{red}{\term[Pytest]{Pytest}} testing functions for the modules
contained within each of the 5 main components of the project. FlagGenerator,
GUI, HashGenerator, HashToFlag, and JKAReader will each have a corresponding
\textcolor{red}{\term[Python]{Python}} \textcolor{red}{\term[Pytest]{Pytest}}
file. Within each main component, most but not all internal functions are able
to be unit tested. The others will require some manual testing. This is
especially evident in the testing of the \textcolor{red}{\term[GUI]{GUI}},
which will need to be done by the user to check functionality and usability
requirements. \textcolor{red}{The remaining Display, Gallery, Settings
and Help modules all need to be tested manually, since they are interaction-based
\term[GUI]{GUI} modules.}\\

\noindent For the modules that are able to be unit tested (modules mentioned
in 3.1.2), assert statements will be setup to check that the modules are
returning the correct values (based on externally calculated values), and that
they are behaving as expected. The ``expected'' cases, boundary/limit cases,
and any exceptions and error handling will be tested through \textcolor{red}{
\term[Pytest]{Pytest}} unit tests. Using the testing coverage matrices, any
unit testable \textcolor{red}{\term[FR]{functional requirements}} will be
covered in the unit tests. Use of external libraries (PIL, tkinter) will have
less thorough testing, since they are commercially available software products,
and \textcolor{red}{are} assumed to have some level of correctness and
\textcolor{red}{testing} already done \textcolor{red}{\sout{testing}}.

\subsection{Unit testing of output files}

\noindent \textcolor{red}{All \sout{Some}} aspects of the outputted flag image
will be able to be validated through unit testing, \textcolor{red}{including
the generated flag image files (which will be tested by comparing the pixel
map colour data of the generated flag image with the expected pixel map colour
data) \sout{but others will be a visual/manual check}}. \textcolor{red}{
Generated flag \sout{Flag generated}} images will have their width and height
checked to ensure the user's resolution selection matches the output image
resolution. Checking known hashing function outputs to make sure the
corresponding flag design (colours, symbols, stripes, etc.) are all correct,
will be done to verify some correctness in the generated symbols, although it
will be unreasonable to find hashes for every single combination of possible
output designs. Unit tests can also be done to check the naming of files, file
storage, and gallery display correctness.

\bibliographystyle{plainnat}

\bibliography{SRS}

\newpage

\section{Appendix}

\subsection{Symbolic Parameters}
The definition of the test cases will call for SYMBOLIC\_CONSTANTS. Their values are defined in this section for easy maintenance. \\ \\
\textcolor{red}{$\hypertarget{min_age}{MIN\_AGE}$ = 7} \\
\textcolor{red}{$\hypertarget{max_find_time}{MAX\_FIND\_TIME}$ = 5} \\
\textcolor{red}{$\hypertarget{max_load_time}{MAX\_LOAD\_TIME}$ = 2} \\

\subsection{Usability Survey Questions\textcolor{red}{\sout{?}}}

Given that the Flag Generator program heavily relies on user input and their
interactions with the GUI, as well as their opinion on the outputs of the
program, survey questions to gauge user experience will be used.

\begin{enumerate}
    \item On a scale of 1-10, how easy was it for your to navigate the program
    using the GUI?
    \item On a scale of 1-10, how appealing/aesthetically pleasing would you
    consider the flag images being generated?
    \item On a scale of 1-10, how straightforward was it to follow the
    intended flow of operations? (ie. entering input $\rightarrow$ accessing generated
    flag image)
    \item On a scale of 1-10, how helpful was the ``Help'' button feature?
    \item Would you like more (or less) ability to set certain parts of the
    flag?
    \item Do you like having the flag gallery view? Is it worthwhile in your
    opinion?
    \item What other designs would you like to see available for the flags
    being generated?
    \item What parts of the GUI were not clear or straightforward to use?
\end{enumerate}

\end{document}