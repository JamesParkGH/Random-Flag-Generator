\documentclass[12pt, titlepage]{article}

\usepackage{fullpage}
\usepackage[round]{natbib}
\usepackage{multirow}
\usepackage{booktabs}
\usepackage{tabularx}
\usepackage{graphicx}
\usepackage{float}
\usepackage{hyperref}
\usepackage[normalem]{ulem}
\hypersetup{
    colorlinks,
    citecolor=black,
    filecolor=black,
    linkcolor=red,
    urlcolor=blue
}
\usepackage[round]{natbib}
\usepackage{xifthen}

\newcounter{acnum}
\newcommand{\actheacnum}{AC\theacnum}
\newcommand{\acref}[1]{AC\ref{#1}}

\newcounter{ucnum}
\newcommand{\uctheucnum}{UC\theucnum}
\newcommand{\uref}[1]{UC\ref{#1}}

\newcounter{mnum}
\newcommand{\mthemnum}{M\themnum}
\newcommand{\mref}[1]{M\ref{#1}}
\newcommand{\newterm}[1]{\label{Term:#1} \MakeUppercase #1}
\newcommand{\term}[2][]{\ifthenelse{\equal{#1}{}}{\hyperref[Term:#2]{\textbf{#2}}}{\hyperref[Term:#1]{\textbf{#2}}}}

\title{SE 3XA3: Module Guide\\Random Flag Generator}

\author{Team \#2, Team Jakriel
		\\ Akram Hannoufa, hannoufa
		\\ Ganghoon (James) Park, parkg10
		\\ Nathaniel Hu, hun4
}

\date{\today}

% \input{../../Comments}

\begin{document}

\maketitle

\pagenumbering{roman}
\tableofcontents
\listoftables
\listoffigures

\begin{table}[bp!]
\caption{\bf Revision History}
\begin{tabularx}{\textwidth}{p{3cm}p{2cm}X}
\toprule {\bf Date} & {\bf Version} & {\bf Notes}\\
\midrule
March 15, 2022 & 1.0 & Initial Document\\
March 16, 2022 & 1.1 & Added initial drafts of Module Hierarchy and Module Decomposition Sections\\
March 17, 2022 & 1.2 & Added initial draft of Anticipated and Unlikely Changes Section; added uses hierarchy diagram to Use Hierarchy Between Modules Section; added traceability matrices to Traceability Matrix Section\\
March 17, 2022 & 1.3 & Added and modified Intro and Connection sections.\\
\textcolor{red}{April 11/12, 2022} & \textcolor{red}{2.0} & \textcolor{red}{Updated Module Guide documentation for Revision 1 submission}\\
\bottomrule
\end{tabularx}
\end{table}

\newpage

\pagenumbering{arabic}

\section{Introduction}

\textcolor{red}{The Random} Flag Generator \term[RFG]{\textcolor{red}{(RFG)}}
is an implementation \textcolor{red}{based on \sout{of}} the open-source
project \term[PAGAN]{\textcolor{red}{PAGAN}}, \textcolor{red}{(Python Avatar
Generation for Absolute Nerds)}. The main functionality of the program comes
from generating a \textcolor{red}{hash digest \sout{hashcode}} from a user's
input string and creating a unique graphical output based on the generated
\textcolor{red}{hash digest \sout{hashcode}}. \textcolor{red}{The
\term[RFG]{RFG} \sout{Flag Generator}} generates a unique flag with its own
set of colours, symbols, designs, etc. \textcolor{red}{The user can set the
flag to be generated entirely based on the hash digest generated from the
user's input string, or choose to set certain elements of the flag (i.e.
choose certain colours or symbols to be included on the generated flag)}.

\subsection{Purpose}
The purpose of the Module Guide document is to outline how this system is
modularly decomposed. Following a module decomposition approach, the
information hiding principle is being adhered to~\citep{Parnas1972a}.
Following this principle also allows new project members, maintainers, and
designers to be able to easily identify the individual components of the
software, derive an understanding of the hierarchical structure of the system,
and to easily verify certain software criteria such as consistency,
flexibility, and feasibility. Additionally, by breaking up the system into
modules, the principles of low cohesion and high coupling.

\subsection{Scope}
The modules used to develop this program were based off of the requirements
outlined in the Software Requirements Specification. The modules' external
behaviours are explicitly described and outlined in the corresponding Module
Interface Specification document. Through adhering to the principle of
information hiding, anticipated and unlikely changes are represented by the
secrets of the modules, allowing for easier changes to be made in the future
(design for change). Lastly, a uses hierarchy is established to graphically
show the relationships between the modules.

\newpage
\subsection{Acronyms, Abbreviations, and Symbols}
    
\begin{table}[htbp]
\caption{\textbf{Table of Abbreviations}} \label{abbrev}

\begin{tabularx}{\textwidth}{p{3cm}X}
\toprule
\textbf{Abbreviation} & \textbf{Definition} \\
\midrule
\newterm{FR} & Functional Requirement\\
\hline
\newterm{GUI} & Graphical User Interface\\
\hline
\newterm{PAGAN} & \term{Python Avatar Generator for Absolute Nerds}\\
\hline
\textcolor{red}{\newterm{RFG}} & \textcolor{red}{\term{Random Flag Generator}}\\
\hline
\newterm{RGB} & Red Green Blue\\
\hline
\newterm{SRS} & Software Requirements Specification\\
\hline
\newterm{UI} & User Interface\\
\bottomrule
\end{tabularx}

\end{table}

\newpage
\begin{table}[htbp]
\caption{\textbf{Table of Definitions}} \label{def}

\begin{tabularx}{\textwidth}{p{3cm}X}
\toprule
\textbf{Term} & \textbf{Definition}\\
\midrule
\newterm{Gallery} & Collection of previously generated flags.\\
\hline
\newterm{Graphical User Interface} & A form of UI that allows users to use
  electronic devices using interactive graphics.\\
\hline
\newterm{Hashing} & Algorithm that converts input data to a fixed-size value.
  A hashing function usually outputs a string or hexadecimal value.\\
\hline
\newterm{Input String} & The input of type string from the user.\\
\hline
\newterm{Pytest} & Python testing tool that allows testers to write test code
  and create simple and scalable test cases.\\
\hline
\newterm{Python} & The programming language used in this project.\\
\hline
\newterm{Software Requirements Specification} & A document that details what
  the program/software will do and how it will accomplish the expected
  performance/tasks.\\
\hline
\newterm{System/Program} & Collection of instructions or components that tell
  a computer how to operate.\\
\hline
\newterm{Tester} & An individual testing the software via the user interface
  or the code/test cases.\\
\hline
\newterm{Typeform} & Website that is a software as a service that specializes
  in creating and building online surveys.\\
\hline
\newterm{User} & Person who uses or operates a computer program.\\
\hline
\newterm{User Interface} & Where interactions between machines and humans
  occur.\\
\bottomrule
\end{tabularx}
    
\end{table} 

\newpage
\subsection{Overview}
The rest of the document is organized as follows. 
\begin{itemize}
  \item Section \ref{SecChange}: Lists the anticipated and unlikely changes of
  the software requirements.
  \item Section \ref{SecMH}: Summarizes the module decomposition that was
  constructed according to the likely changes.
  \item Section \ref{SecConnection}: Specifies the connections between the
  software requirements and the modules.
  \item Section \ref{SecMD}: Gives a detailed description of the modules.
  \item Section \ref{SecTM}: Includes two traceability matrices. One checks
  the completeness of the design against the requirements provided in the SRS.
  The other shows the relation between anticipated changes and the modules.
  \item Section \ref{SecUse}: Describes the use relation between modules.
\end{itemize}

\section{Anticipated and Unlikely Changes} \label{SecChange}

This section lists possible changes to the system. According to the likeliness
of the change, the possible changes are classified into two categories.
Anticipated changes are listed in Section \ref{SecAchange}, and unlikely
changes are listed in Section \ref{SecUchange}.

\subsection{Anticipated Changes} \label{SecAchange}
Anticipated changes are the source of the information that is to be hidden
inside the modules. Ideally, changing one of the anticipated changes will only
require changing the one module that hides the associated decision. The
approach adapted here is called design for change.

\begin{description}
\item[\refstepcounter{acnum} \actheacnum \label{ac1}:] The addition of new
  flag assets.
\item[\refstepcounter{acnum} \actheacnum \label{ac2}:] The flag resolution
  options.
\item[\refstepcounter{acnum} \actheacnum \label{ac3}:] Available hashing
  functions.
\item[\refstepcounter{acnum} \actheacnum \label{ac4}:] Maximum size of flag
  gallery(number of images).
\item[\refstepcounter{acnum} \actheacnum \label{ac5}:] Where and how flags are
  displayed after being generated.
\item[\refstepcounter{acnum} \actheacnum \label{ac6}:] More user control over
  user input streams; output.
\item[\refstepcounter{acnum} \actheacnum \label{ac7}:] Output file format.
\item[\refstepcounter{acnum} \actheacnum \label{ac8}:] Input string criteria,
  wider range of allowable characters (broader input range).
\item[\refstepcounter{acnum} \actheacnum \label{ac9}:] Making changes to how
  to use the program.
\end{description}

\subsection{Unlikely Changes} \label{SecUchange}
The module design should be as general as possible. However, a general system
is more complex. Sometimes this complexity is not necessary. Fixing some
design decisions at the system architecture stage can simplify the software
design. If these decision should later need to be changed, then many parts of
the design will potentially need to be modified. Hence, it is not intended
that these decisions will be changed.

\begin{description}
\item[\refstepcounter{ucnum} \uctheucnum \label{uc1}:] The same graphics
  library (PIL/pillow) will be used to generate the flag image files.
\item[\refstepcounter{ucnum} \uctheucnum \label{uc2}:] The same asset reader
  (JKAReader) will be used to read the flag symbol design.
\item[\refstepcounter{ucnum} \uctheucnum \label{uc3}:] String input will be
  used to generate the hash digests.
\item[\refstepcounter{ucnum} \uctheucnum \label{uc4}:] Input/output devices
  (input: keyboard, output: monitor and file).
\item[\refstepcounter{ucnum} \uctheucnum \label{uc5}:] The same hashing
  library (hashlib) will be used to get the hashing functions.
\end{description}

\section{Module Hierarchy} \label{SecMH}

This section provides an overview of the module design. Modules are summarized
in a hierarchy decomposed by secrets in Table \ref{TblMH}. The modules listed
below, which are leaves in the hierarchy tree, are the modules that will
actually be implemented.

\begin{description}
\item [\refstepcounter{mnum} \mthemnum \label{M1}:] HashGenerator Module
\item [\refstepcounter{mnum} \mthemnum \label{M2}:] FlagAssetsLib Module
\item [\refstepcounter{mnum} \mthemnum \label{M3}:] JKAReader Module
\item [\refstepcounter{mnum} \mthemnum \label{M4}:] DecisionUtilities Module
\item [\refstepcounter{mnum} \mthemnum \label{M5}:] HashToFlag Module
\item [\refstepcounter{mnum} \mthemnum \label{M6}:] FlagGenerator Module
\item [\refstepcounter{mnum} \mthemnum \label{M7}:] GUI Module
\item [\refstepcounter{mnum} \mthemnum \label{M8}:] Display Module
\item [\refstepcounter{mnum} \mthemnum \label{M9}:] Gallery Module
\item [\refstepcounter{mnum} \mthemnum \label{M10}:] \textcolor{red}{Help
  \sout{HelpInstructions}} Module
\item [\refstepcounter{mnum} \mthemnum \label{M11}:] Settings Module
\end{description}


\begin{table}[h!]
\centering
\begin{tabular}{p{0.3\textwidth} p{0.6\textwidth}}
\toprule
\textbf{Level 1} & \textbf{Level 2}\\
\midrule

{Hardware-Hiding Module} & ~ \\
\midrule

\multirow{7}{0.3\textwidth}{Behaviour-Hiding Module} & HashGenerator \\
& FlagAssetsLib \\
& JKAReader \\
& GUI \\
& Display \\
& Gallery \\ 
& \textcolor{red}{Help \sout{HelpInstructions}} \\
& Settings \\
\midrule

\multirow{3}{0.3\textwidth}{Software Decision Module} & DecisionUtilities \\
& HashToFlag \\
& FlagGenerator \\
\bottomrule

\end{tabular}
\caption{Module Hierarchy}
\label{TblMH}
\end{table}

\section{Connection Between Requirements and Design} \label{SecConnection}

The design of the system is intended to satisfy the requirements developed in
the \textcolor{red}{\term[SRS]{SRS}}. In this stage, the system is decomposed
into modules. The connection between requirements and modules is listed in
Table \ref{TblRT}, in the form of a traceability matrix. 

Every \textcolor{red}{\term[FR]{functional requirement}} is implemented by at
least one module, and similarly the modules responsible for handling the
anticipated changes are also laid out in a traceability matrix, Table
\ref{TblACT}.

Modules are generally broken up into hash \textcolor{red}{digest} generation,
\textcolor{red}{hash code \sout{hashcode}} to flag decisions and
\textcolor{red}{\term[GUI]{GUI}} modules. Within these categories, there is a
breakdown of modules to handle specific functionalities. The \textcolor{red}{
\term[GUI]{GUI}} functionality is broken down into \textcolor{red}{the}
Display, Gallery, Help, and Settings modules to handle specific
\textcolor{red}{\term[FR]{functional requirements}}. Similarly, HashToFlag's
functionality is broken down into \textcolor{red}{the \sout{a}} FlagAssetsLib
module, the HashGenerator module (responsible for generating the
\textcolor{red}{hash digest \sout{hashcode}}), and \textcolor{red}{the
\sout{a}} DecisionUtilities module (with helper functions for converting a
\textcolor{red}{hash digest \sout{hashcode}} to decisions). Lastly, the
JKAReader module (responsible for converting .jka files into pixels) combined
with the HashToFlag module comprises the majority of the FlagGenerator
functionality.

\section{Module Decomposition} \label{SecMD}

Modules are decomposed according to the principle of ``information hiding''
proposed by \citet{ParnasEtAl1984}. The \emph{Secrets} field in a module
decomposition is a brief statement of the design decision hidden by the
module. The \emph{Services} field specifies \emph{what} the module will do
without documenting \emph{how} to do it. For each module, a suggestion for the
implementing software is given under the \emph{Implemented By} title. If the
entry is \emph{OS}, this means that the module is provided by the operating
system or by standard programming language libraries.  

\subsection{Hardware Hiding Modules}

N/A

% \begin{description}
% \item[Secrets:]The data structure and algorithm used to implement the virtual
%   hardware.
% \item[Services:]Serves as a virtual hardware used by the rest of the
%   system. This module provides the interface between the hardware and the
%   software. So, the system can use it to display outputs or to accept inputs.
% \item[Implemented By:] OS
% \end{description}

\subsection{Behaviour-Hiding Module}

\subsubsection{HashGenerator Module (\mref{M1})}

\begin{description}
\item[Secrets:] Hashing algorithms available and method for generating hash
  digests from input strings and selected hashing algorithm (if available)
\item[Services:] Generates hash digests from input strings using selected
  hashing algorithms (if available; default algorithm is SHA-256)
\item[Implemented By:] Random Flag Generator
\end{description}

\subsubsection{FlagAssetsLib Module (\mref{M2})}

\begin{description}
\item[Secrets:] Flag assets, constants and values used to generate flags
\item[Services:] Provides the data files and values needed to generate flags
\item[Implemented By:] Random Flag Generator
\end{description}

\subsubsection{JKAReader Module (\mref{M3})}

\begin{description}
\item[Secrets:] Method for parsing flag asset (.jka) file data into usable
  form for generating flags
\item[Services:] Parses flag asset (.jka) file data into pixel maps to be used
  to generate flags
\item[Implemented By:] Random Flag Generator
\end{description}

\subsubsection{GUI Module (\mref{M7})}

\begin{description}
\item[Secrets:] Methods for starting the program and presenting the
  \textcolor{red}{\term[GUI]{GUI}} view, taking in user input strings and
  selected settings for generating flags
\item[Services:] Starts the program and presents the \textcolor{red}{
  \term[GUI]{GUI}} view, takes in user input strings and selected settings for
  generating flags
\item[Implemented By:] Random Flag Generator
\end{description}

\subsubsection{Display Module (\mref{M8})}

\begin{description}
\item[Secrets:] Methods for displaying the associated generated flags
\item[Services:] Displays the associated generated flags via a
  \textcolor{red}{\term[GUI]{GUI}}
\item[Implemented By:] Random Flag Generator
\end{description}

\subsubsection{Gallery Module (\mref{M9})}

\begin{description}
\item[Secrets:] Saved flags in user's flag \textcolor{red}{
  \term[Gallery]{gallery}} and methods for displaying all saved flags from the
  user's flag \textcolor{red}{\term[Gallery]{gallery} \sout{and searching up
  flags by name (i.e. input string)}}
\item[Services:] Displays all the saved flags from the user's flag
  \textcolor{red}{\term[Gallery]{gallery}} via an interactive \textcolor{red}{
  \term[GUI]{GUI}}
\item[Implemented By:] Random Flag Generator
\end{description}

\subsubsection{\textcolor{red}{Help \sout{HelpInstructions}} Module (\mref{M10})}

\begin{description}
\item[Secrets:] Help and instructions information and methods for displaying
  the help and instructions menu
\item[Services:] Displays the helps and instructions menu and information via
  a \textcolor{red}{\term[GUI]{GUI}}
\item[Implemented By:] Random Flag Generator
\end{description}

\subsubsection{Settings Module (\mref{M11})}

\begin{description}
\item[Secrets:] \textcolor{red}{Methods \sout{Program version and methods}}
  for displaying the settings menu, taking in user settings changes and
  updating settings
\item[Services:] Displays the settings menu \textcolor{red}{\sout{and program
  version}} via an interactive \textcolor{red}{\term[GUI]{GUI}}
\item[Implemented By:] Random Flag Generator
\end{description}

\subsection{Software Decision Module}

\subsubsection{DecisionUtilities Module (\mref{M4})}

\begin{description}
\item[Secrets:] Auxiliary helper methods for padding generated hash digests
  (that are too short) and assisting in flag generation information parsing
  and usage in generating flags, and associated constants and values that will
  be used by the auxiliary helper methods
\item[Services:] Pads generated hash digests (that are too short) and assists
  in flag generation information parsing and usage in generating flags
\item[Implemented By:] Random Flag Generator
\end{description}

\subsubsection{HashToFlag Module (\mref{M5})}

\begin{description}
\item[Secrets:] Methods for grinding generated hash digests to select the
  colours and flag design elements to generate flags
\item[Services:] Provides the selected colours and flag design elements
  obtained from grinding the generated hash digest to generate flags
\item[Implemented By:] Random Flag Generator
\end{description}

\subsubsection{FlagGenerator Module (\mref{M6})}

\begin{description}
\item[Secrets:] Methods for obtaining the colours and flag design elements and
  generating the flags and saving the flag image files to the flag
  \textcolor{red}{\term[Gallery]{gallery}}
\item[Services:] Obtains the colours and flag design elements and generates
  the flags and saves the flag image files to the flag \textcolor{red}{
  \term[Gallery]{gallery}}
\item[Implemented By:] Random Flag Generator
\end{description}

\section{Traceability Matrix} \label{SecTM}

This section shows two traceability matrices: between the modules and the
requirements and between the modules and the anticipated changes.

% the table should use mref, the requirements should be named, use something
% like fref
\begin{table}[H]
\centering
\begin{tabular}{p{0.2\textwidth} p{0.6\textwidth}}
\toprule
\textbf{Req.} & \textbf{Modules}\\
\midrule
FR1 & \mref{M7}\\
FR2 & \mref{M7}\\
FR3 & \mref{M7}\\
FR4 & \mref{M1}, \mref{M2}, \mref{M3}, \mref{M4}, \mref{M5}, \mref{M6}, \mref{M7}\\
FR5 & \mref{M8}\\
FR6 & \mref{M1}, \mref{M2}, \mref{M3}, \mref{M4}, \mref{M5}, \mref{M6}, \mref{M11}\\
FR7 & \mref{M6}\\
FR8 & \mref{M1}, \mref{M2}, \mref{M3}, \mref{M4}, \mref{M5}, \mref{M6}, \mref{M7}\\
FR9 & \mref{M8}\\
FR10 & \mref{M1}, \mref{M2}, \mref{M3}, \mref{M4}, \mref{M5}, \mref{M6}, \mref{M11}\\
FR11 & \mref{M6}, \mref{M7}, \mref{M9}\\
FR12 & \mref{M10}\\
FR13 & \mref{M10}\\
FR14 & \mref{M11}\\
FR15 & \mref{M11}\\
FR16 & \mref{M11}\\
FR17 & \mref{M9}\\
FR18 & \mref{M9}\\
FR19 & \mref{M9}\\
\bottomrule
\end{tabular}
\caption{Trace Between Requirements and Modules}
\label{TblRT}
\end{table}

\begin{table}[H]
\centering
\begin{tabular}{p{0.2\textwidth} p{0.6\textwidth}}
\toprule
\textbf{AC} & \textbf{Modules}\\
\midrule
\acref{ac1} & \mref{M2}, \mref{M3}\\
\acref{ac2} & \mref{M2}\\
\acref{ac3} & \mref{M1}\\
\acref{ac4} & \mref{M9}\\
\acref{ac5} & \mref{M8}\\
\acref{ac6} & \mref{M4}, \mref{M5}, \mref{M6}, \mref{M7}, \mref{M11}\\
\acref{ac7} & \mref{M6}\\
\acref{ac8} & \mref{M1}, \mref{M7}\\
\acref{ac9} & \mref{M10}\\
\bottomrule
\end{tabular}
\caption{Trace Between Anticipated Changes and Modules}
\label{TblACT}
\end{table}

\section{Use Hierarchy Between Modules} \label{SecUse}

In this section, the uses hierarchy between modules is provided.
\citet{Parnas1978} said of two programs A and B that A {\em uses} B if correct
execution of B may be necessary for A to complete the task described in its
specification. That is, A {\em uses} B if there exist situations in which the
correct functioning of A depends upon the availability of a correct
implementation of B.  Figure \ref{FigUH} illustrates the use relation between
the modules. It can be seen that the graph is a directed acyclic graph (DAG).
Each level of the hierarchy offers a testable and usable subset of the system,
and modules in the higher level of the hierarchy are essentially simpler
because they use modules from the lower levels.

\begin{figure}[H]
\centering
\includegraphics[width=1.0\textwidth]{UsesHierarchy.png}
\caption{Use hierarchy among modules}
\label{FigUH}
\end{figure}


\bibliographystyle {plainnat}
\bibliography {MG}

\end{document}