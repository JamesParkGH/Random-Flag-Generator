\documentclass[12pt, titlepage]{article}
\usepackage[margin=1in]{geometry}
\usepackage{booktabs}
\usepackage{tabularx}
\usepackage{hhline}
\usepackage{lscape}
\usepackage{hyperref}
\hypersetup{
    colorlinks,
    citecolor=black,
    filecolor=black,
    linkcolor=black,
    urlcolor=blue
}
\usepackage[round]{natbib}
\usepackage{titlesec}
\usepackage{placeins}
\usepackage{graphicx}
\usepackage{xkeyval}
\usepackage[dvipsnames]{xcolor} % for different colour comments
\usepackage{tabto}
\usepackage{mdframed}
\usepackage{lscape}
\usepackage{multirow}


\title{SE 3XA3: Test Report\\Random Flag Generator}

\author{Team \#2, Team Jakriel
		\\ Nathaniel Hu, hun4
		\\ Ganghoon Park, parkg10
		\\ Akram Hannoufa, hannoufa
}

\date{\today}

\makeatletter

\define@cmdkey      [TP] {test}     {name}       {}
\define@cmdkey      [TP] {test}     {desc}       {}
\define@cmdkey      [TP] {test}     {type}       {}
\define@cmdkey      [TP] {test}     {init}       {}
\define@cmdkey      [TP] {test}     {input}      {}
\define@cmdkey      [TP] {test}     {output}     {}
\define@cmdkey      [TP] {test}     {pass}       {}
\define@cmdkey      [TP] {test}     {user}       {}
\define@cmdkey      [TP] {test}     {result}     {}


\newcommand{\getCurrentSectionNumber}{%
  \ifnum\c@section=0 %
  \thechapter
  \else
  \ifnum\c@subsection=0 %
  \thesection
  \else
  \ifnum\c@subsubsection=0 %
  \thesubsection
  \else
  \thesubsubsection
  \fi
  \fi
  \fi
}
\newcounter{TestNum}


\newcommand{\testauto}[1]{
\setkeys[TP]{test}{#1}
\refstepcounter{TestNum}
\begin{mdframed}[linewidth=1pt]
\begin{tabularx}{\textwidth}{@{}p{3cm}X@{}}
{\bf Test \#\theTestNum:} & {\bf \cmdTP@test@name}\\[\baselineskip]
{\bf Description:} & \cmdTP@test@desc\\[0.5\baselineskip]
{\bf Type:} & \cmdTP@test@type\\[0.5\baselineskip]
{\bf Initial State:} & \cmdTP@test@init\\[0.5\baselineskip]
{\bf Input:} & \cmdTP@test@input\\[0.5\baselineskip]
{\bf Output:} & \cmdTP@test@output\\[0.5\baselineskip]
{\bf Expected:} & \cmdTP@test@pass\\[\baselineskip]
{\bf Result:} & \cmdTP@test@result
\end{tabularx}
\end{mdframed}
}

\newcommand{\testautob}[1]{
\setkeys[TP]{test}{#1}
\refstepcounter{TestNum}
\begin{mdframed}[linewidth=1pt]
\begin{tabularx}{\textwidth}{@{}p{3cm}X@{}}
{\bf Test \#\theTestNum:} & {\bf \cmdTP@test@name}\\[\baselineskip]
{\bf Description:} & \cmdTP@test@desc\\[0.5\baselineskip]
{\bf Type:} & \cmdTP@test@type\\[0.5\baselineskip]
{\bf Pass:} & \cmdTP@test@pass\\[\baselineskip]
{\bf Result:} & \cmdTP@test@result
\end{tabularx}
\end{mdframed}
}

\newcommand{\testmanual}[1]{
\setkeys[TP]{test}{#1}
\refstepcounter{TestNum}
\begin{mdframed}[linewidth=1pt]
\begin{tabularx}{\textwidth}{@{}p{3cm}X@{}}
{\bf Test \#\theTestNum:} & {\bf \cmdTP@test@name}\\[\baselineskip]
{\bf Description:} & \cmdTP@test@desc\\[0.5\baselineskip]
{\bf Type:} & \cmdTP@test@type\\[0.5\baselineskip]
{\bf Tester(s):} & \cmdTP@test@user\\[0.5\baselineskip]
{\bf Pass:} & \cmdTP@test@pass\\[\baselineskip]
{\bf Result:} & \cmdTP@test@result
\end{tabularx}
\end{mdframed}
}

\usepackage{xifthen}
\def\namedlabel#1#2{\begingroup
    #2%
    \def\@currentlabel{#2}%
    \phantomsection\label{#1}\endgroup
}


\newcommand{\newterm}[1]{\label{Term:#1} \MakeUppercase #1}
\newcommand{\term}[2][]{\ifthenelse{\equal{#1}{}}{\hyperref[Term:#2]{\textbf{#2}}}{\hyperref[Term:#1]{\textbf{#2}}}}




\begin{document}

\maketitle

\pagenumbering{roman}
\tableofcontents
\listoftables
\listoffigures

\newpage
\begin{table}[hbp!]
\caption{\bf Revision History}
\begin{tabularx}{\textwidth}{p{3cm}p{2cm}X}
\toprule {\bf Date} & {\bf Version} & {\bf Notes}\\
\midrule
2022-04-11 & 1.0 & Initial document creation\\
2022-04-12 & 1.1 & Added sections 1, 3, 4, 5 and 6\\
2022-04-12 & 1.2 & Cleaned up formatting of revision history, tables of abbreviations and definitions and document introduction\\
2022-04-12 & 1.3 & Added initial draft of section 2 non-functional requirements evaluation, as well as symbolic parameters subsection and uses\\
2022-04-12 & 1.4 & Added initial draft of section 7 trace to requirements\\
2022-04-12 & 1.5 & Added initial draft of section 8 trace to modules\\
\textcolor{red}{2022-04-12} & \textcolor{red}{2.0} & \textcolor{red}{Completed Test Report documentation for Revision 1 submission}\\
\bottomrule
\end{tabularx}
\end{table}

\newpage

\pagenumbering{arabic}
\newpage
    
\begin{table}[htbp]
\caption{\textbf{Table of Abbreviations}} \label{abbrev}

\begin{tabularx}{\textwidth}{p{3cm}X}
\toprule
\textbf{Abbreviation} & \textbf{Definition} \\
\midrule
\newterm{FR} & Functional Requirement\\
\hline
\newterm{GUI} & Graphical User Interface\\
\hline
\newterm{LF} & Look and Feel test\\
\hline
\newterm{NFR} & \term{Non-Functional Requirement}\\
\hline
\newterm{PAGAN} & \term{Python Avatar Generator for Absolute Nerds}\\
\hline
\newterm{PE} & Performance test\\
\hline
\newterm{RFG} & \term{Random Flag Generator}\\
\hline
\newterm{RGB} & Red Green Blue\\
\hline
\newterm{SRS} & Software Requirements Specification\\
\hline
\newterm{UH} & Usability and Humanity test\\
\hline
\newterm{UI} & User Interface\\
\bottomrule
\end{tabularx}

\end{table}

\newpage
\begin{table}[htbp]
\caption{\textbf{Table of Definitions}} \label{def}

\begin{tabularx}{\textwidth}{p{3cm}X}
\toprule
\textbf{Term} & \textbf{Definition}\\
\midrule
\newterm{Gallery} & Collection of previously generated flags.\\
\hline
\newterm{Graphical User Interface} & A form of UI that allows users to use
  electronic devices using interactive graphics.\\
\hline
\newterm{Hashing} & Algorithm that converts input data to a fixed-size value.
  A hashing function usually outputs a string or hexadecimal value.\\
\hline
\newterm{Input String} & The input of type string from the user.\\
\hline
\newterm{Pytest} & Python testing tool that allows testers to write test code
  and create simple and scalable test cases.\\
\hline
\newterm{Python} & The programming language used in this project.\\
\hline
\newterm{Software Requirements Specification} & A document that details what
  the program/software will do and how it will accomplish the expected
  performance/tasks.\\
\hline
\newterm{System/Program} & Collection of instructions or components that tell
  a computer how to operate.\\
\hline
\newterm{Tester} & An individual testing the software via the user interface
  or the code/test cases.\\
\hline
\newterm{Test Case} & a specification of inputs, execution conditions,
  producedure and expected results for testing a program's behaviour.\\
\hline
\newterm{Typeform} & Website that is a software as a service that specializes
  in creating and building online surveys.\\
\hline
\newterm{User} & Person who uses or operates a computer program.\\
\hline
\newterm{User Interface} & Where interactions between machines and humans
  occur.\\
\bottomrule
\end{tabularx}
    
\end{table} 

\newpage

This document details the complete testing process for Random Flag Generator,
as laid out in the project test plan. It contains an evaluation of the
project's functional and non-functional requirements that are defined in the
\hyperlink{https://gitlab.cas.mcmaster.ca/palanicv/3xa3__l01_gr16_project/-/blob/master/Doc/Rev1/SRS/RequirementsDocument.pdf}{\textbf{Software Requirements Specification}},
the changes made due to testing, and an analysis of the traceability between
requirements and modules.

\section{Functional Requirements Evaluation}
The following are the test cases that were evaluated for testing the
functional requirements of this system.

\subsection{User Interface Testing}
\testauto{
    name = {FR-01},
    desc =  UI initialization,
    type = {Functional, Dynamic, Manual},
    init = UI is uninitialized,
    input = command to initialize and run the UI,
    output = UI is initialized,
    pass = {main menu UI should open and allow the
    user to enter text into the input string field. The main menu should also
    present buttons for the instructions, flag gallery and settings menus},
    result = \textcolor{Green}{PASS}
}
\testauto{
    name = {FR-02},
    desc =  UI handles button clicks,
    type = {Functional, Dynamic, Manual},
    init = UI is initialized and running,
    input = {mouse clicks on the instructions, flag gallery or settings menu
    buttons, or the appropriate keyboard strokes assigned to open each menu},
    output = {the instructions, flag gallery or settings menu will be pulled up and
    can be viewed in its entirety},
    pass = {From main menu UI, button clicks pull up appropriate menu},
    result = \textcolor{Green}{PASS}
}
\testauto{
    name = {FR-03},
    desc =  UI handles user text input,
    type = {Functional, Dynamic, Manual},
    init = UI is initialized and running,
    input = {keyboard strokes to (re)enter text into the input string field},
    output = {text appears in the input string field},
    pass = {User is able to enter and re-enter input strings and it shows up in the input string field},
    result = \textcolor{Green}{PASS}
}
\testauto{
    name = {FR-04-08},
    desc =  flag is generated and saved,
    type = {Functional, Dynamic, Manual},
    init = UI is initialized and running,
    input = {(text in input string field and) mouse click on generate flag button},
    output = {flag will be generated using the input string and saved on the local
    machine (in \/generated\_flags directory)},
    pass = {Clicking the generate button, generates and saves the flag image},
    result = \textcolor{Green}{PASS}
}
\testauto{
    name = {FR-05-09},
    desc =  flag is displayed,
    type = {Functional, Dynamic, Manual},
    init = {UI is initialized and running, a flag has already finished
    generating},
    input = {(text in the input string field and) mouse click on the display flag
    button.},
    output = {generated flag with be displayed to the user through the UI},
    pass = {Clicking the display button, displays the most recently generated flag image},
    result = \textcolor{Green}{PASS}
}
\testauto{
    name = {FR-06-10},
    desc =  flag settings overrides the most recent display,
    type = {Functional, Dynamic, Manual},
    init = {UI is initialized and running, a flag has already finished generating},
    input = {(text in the input string field,) saved changes to the settings and
    mouse click on the display flag button},
    output = {flag will be regenerated using the input string and changed settings,
    and saved on the local machine (in \/generated\_flags directory)},
    pass = {User selected settings overrides the most recently generated flag},
    result = \textcolor{Green}{PASS}
}
\testauto{
    name = {FR-07-11},
    desc =  newly generated flag should appear in flag gallery,
    type = {Functional, Dynamic, Manual},
    init = {UI is initialized and running},
    input = {(text in the input string field and) mouse click on the generate flag button},
    output = {flag will be generated using the input string and saved on the local
    machine (in \/generated\_flags directory) using the input string as its (default) name},
    pass = {User's newly generated flag should appear in flag gallery},
    result = \textcolor{Green}{PASS}
}
\testauto{
    name = {FR-12-13},
    desc =  UI displays instructions menu,
    type = {Functional, Dynamic, Manual},
    init = {UI is initialized and running},
    input = {mouse click on the instructions button, or the appropriate keyboard
    stroke assigned to open the instructions menu},
    output = {the instructions menu will be pulled up and can be viewed in its
    entirety, and a button will be displayed that the user can click on to close
    the instructions menu and return to the main menu},
    pass = {UI displays instructions menu after clicking instructions button and can return to main menu},
    result = \textcolor{Green}{PASS}
}
\testauto{
    name = {FR-14},
    desc =  UI displays settings menu,
    type = {Functional, Dynamic, Manual},
    init = {UI is initialized and running},
    input = {mouse click on the settings button, or the appropriate keyboard stroke
    assigned to open the settings menu, and clicks/dragging to change settings},
    output = {the settings menu will be pulled up where settings can be changed},
    pass = {UI displays settings menu after clicking settings button and can return to main menu},
    result = \textcolor{Green}{PASS}
}
\testauto{
    name = {FR-15},
    desc =  UI displays version number,
    type = {Functional, Dynamic, Manual},
    init = {UI is initialized and running},
    input = {mouse click on the settings button, or the appropriate keyboard stroke
    assigned to open the settings menu},
    output = {the settings menu will be pulled up where the flag generator version
    will be displayed},
    pass = {UI displays version number on settings menu after clicking settings button and can return to main menu},
    result = \textcolor{Green}{PASS}
}
\testauto{
    name = {FR-16},
    desc =  Close settings menu to view main menu,
    type = {Functional, Dynamic, Manual},
    init = {UI is initialized and running},
    input = {mouse click on the settings button, or the appropriate keyboard stroke
    assigned to open the settings menu},
    output = {the settings menu will be pulled up where a return to main menu button
    will be displayed that the user can click on to close the settings menu and
    return to the main menu},
    pass = {From the settings menu, clicking return to main menu closes settings and returns to main menu},
    result = \textcolor{Green}{PASS}
}
\testauto{
    name = {FR-17},
    desc =  Open flag gallery menu from main menu,
    type = {Functional, Dynamic, Manual},
    init = {UI is initialized and running},
    input = {mouse click on the flag gallery button, or the appropriate keyboard
    stroke assigned to open the flag gallery menu},
    output = {the flag gallery menu will be pulled up where a list of all flags and
    the input strings used to generate them will be displayed},
    pass = {From the main menu, clicking on the flag gallery button brings up the flag gallery menu},
    result = \textcolor{Green}{PASS}
}
\testauto{
    name = {FR-19},
    desc =  Return to main menu from flag gallery menu,
    type = {Functional, Dynamic, Manual},
    init = {UI is initialized and running},
    input = {mouse click on the flag gallery button, or the appropriate keyboard
    stroke assigned to open the flag gallery menu},
    output = {the flag gallery menu will be pulled up where a button will be
    displayed that the user can click on to close the flag gallery menu and return
    to the main menu},
    pass = {From the settings menu, clicking return to main menu closes settings and returns to main menu},
    result = \textcolor{Green}{PASS}
}

\subsection{Output Testing}
\testauto{
    name = {test\_get\_hash\_algo},
    desc =  getting the correct hashing algorithm,
    type = {Functional, Dynamic, Automated},
    init = {},
    input = {input string of hashing algorithm name},
    output = {hashlib hashing algorithm},
    pass = {returns the correct hashing algorithm based on the given input},
    result = \textcolor{Green}{PASS}
}
\testauto{
    name = {test\_get\_hash\_hex},
    desc =  getting the correct hash digest from input,
    type = {Functional, Dynamic, Automated},
    init = {},
    input = {input string to be used to generate flag, hashlib hashing algorithm},
    output = {hash digest of input string in hexadecimal form},
    pass = {returns the correct hash digest in hexadecimal form based on the given input},
    result = \textcolor{Green}{PASS}
}
\testauto{
    name = {test\_hash\_generator},
    desc =  getting the correct hash digest from input,
    type = {Functional, Dynamic, Automated},
    init = {},
    input = {input string to be used to generate flag},
    output = {hash digest of input string in hexadecimal form},
    pass = {returns the correct hash digest in hexadecimal form based on the given input},
    result = \textcolor{Green}{PASS}
}
\testauto{
    name = {test\_pad\_hashcode},
    desc =  padding hash digest to minimum length,
    type = {Functional, Dynamic, Automated},
    init = {},
    input = {output hash string to be used to generate flag},
    output = {output hash string padded to minimum hash length},
    pass = {returns a modified hash digest of at least minimum length},
    result = \textcolor{Green}{PASS}
}
\testauto{
    name = {test\_choose\_from\_list},
    desc =  choose element from list,
    type = {Functional, Dynamic, Automated},
    init = {},
    input = {list of elements, float value to determine an index to
    be used to select an element from the list},
    output = {element from the list with index larger than and
    closest to input float value},
    pass = {returns the correct value from the list},
    result = \textcolor{Green}{PASS}
}
\testauto{
    name = {test\_map\_decision},
    desc =  generating an float index value,
    type = {Functional, Dynamic, Automated},
    init = {},
    input = {three numerical values representing the maximum
    possible option, the number of possible decisions and the digit to map within
    the possible options},
    output = {float index value to be used to decide which element
    to select (presumably from a list)},
    pass = {returns the correct float value representing an index},
    result = \textcolor{Green}{PASS}
}
\testauto{
    name = {test\_split\_sequence},
    desc =  split string into substrings,
    type = {Functional, Dynamic, Automated},
    init = {},
    input = {input string, integer value specifying the length of
    the substrings to be split from the input string},
    output = {list of substrings from the input string with as many
    substrings of the specified length as possible},
    pass = {returns a list of substrings of the corrent length},
    result = \textcolor{Green}{PASS}
}
\testauto{
    name = {test\_grind\_hash\_for\_colors},
    desc =  generate colors from hash digest,
    type = {Functional, Dynamic, Automated},
    init = {},
    input = {output hash string to be used to generate flag},
    output = {list of five colors' RGB values (R, G, B)},
    pass = {returns a list of 5 correct RGB values},
    result = \textcolor{Green}{PASS}
}
\testauto{
    name = {test\_grind\_hash\_for\_base\_stripe\_style},
    desc =  generate stripe style from hash digest,
    type = {Functional, Dynamic, Automated},
    init = {},
    input = {output hash string to be used to generate flag},
    output = {output string of base stripe style to be used to
    generate flag},
    pass = {return correct base striple style value},
    result = \textcolor{Green}{PASS}
}
\testauto{
    name = {test\_grind\_hash\_for\_overlay\_stripe\_style},
    desc =  generate stripe style from hash digest,
    type = {Functional, Dynamic, Automated},
    init = {},
    input = {output hash string to be used to generate flag},
    output = {output string of overlay stripe style to be used to
    generate flag},
    pass = {return correct overlay striple style value},
    result = \textcolor{Green}{PASS}
}
\testauto{
    name = {test\_grind\_hash\_for\_stripe\_number},
    desc =  generate stripe number from hash digest,
    type = {Functional, Dynamic, Automated},
    init = {},
    input = {output hash string to be used to generate flag},
    output = {output string of stripe number to be used to
    generate flag},
    pass = {return correct stripe number value},
    result = \textcolor{Green}{PASS}
}
\testauto{
    name = {test\_grind\_hash\_for\_symbol\_locations},
    desc =  generate symbol location from hash digest,
    type = {Functional, Dynamic, Automated},
    init = {},
    input = {output hash string to be used to generate flag},
    output = {output string of symbol location to be used to
    generate flag},
    pass = {return correct symbol location value},
    result = \textcolor{Green}{PASS}
}
\testauto{
    name = {test\_grind\_hash\_for\_symbol\_number},
    desc =  generate symbol number from hash digest,
    type = {Functional, Dynamic, Automated},
    init = {},
    input = {output hash string to be used to generate flag},
    output = {output string of symbol number to be used to
    generate flag},
    pass = {return correct symbol number value},
    result = \textcolor{Green}{PASS}
}
\testauto{
    name = {test\_grind\_hash\_for\_symbol\_types},
    desc =  generate symbol type from hash digest,
    type = {Functional, Dynamic, Automated},
    init = {},
    input = {output hash string to be used to generate flag},
    output = {output string of symbol type to be used to
    generate flag},
    pass = {return correct symbol type value},
    result = \textcolor{Green}{PASS}
}
\testauto{
    name = {test\_hex2rgb},
    desc =  generate RGB value from hexadecimal input,
    type = {Functional, Dynamic, Automated},
    init = {},
    input = {output hexidecimal color code string to be used to generate flag},
    output = {output tuple of RGB values for color derived from heximdecimal color
    code string to be used to generate flag},
    pass = {return corresponding RGB tuple values based on hexadecimal input},
    result = \textcolor{Green}{PASS}
}
\testauto{
    name = {test\_diff},
    desc =  absolute difference of float values,
    type = {Functional, Dynamic, Automated},
    init = {},
    input = {two float values},
    output = {absolute value of the difference between the two given
    float value},
    pass = {return correct absolute difference of the two input floats},
    result = \textcolor{Green}{PASS}
}
\testauto{
    name = {test\_generate\_flag},
    desc =  flag is generated correctly,
    type = {Functional, Dynamic, Automated, Manual},
    init = {},
    input = {input string, hashing algorithm name string and
    dictionary of settings (i.e. chosen colours, elements)to be used to generate
    flag},
    output = {generated flag image file},
    pass = {Manual visual comparison of generated flag agrees with expected output, flag pixels match expected pixels},
    result = \textcolor{Green}{PASS}
}
\testauto{
    name = {test\_generate\_flag\_data},
    desc =  flag data is generated correctly,
    type = {Functional, Dynamic, Automated},
    init = {},
    input = {input string, hashing algorithm name string to be used to generate flag},
    output = {tuple consisting of list of 5 tuples of RGB values, tuple of stripe
    style, number and tuple of symbol location, number and type},
    pass = {All generated outputs match their expected output values},
    result = \textcolor{Green}{PASS}
}
\testauto{
    name = {test\_parse\_jka\_file},
    desc =  asset files are properly read,
    type = {Functional, Dynamic, Automated},
    init = {},
    input = {input string of flag asset (.jka) file name to be
    parsed for pixel data},
    output = {list consisting of tuples of (x, y) coordinates of the
    position of all filled pixels that compose the selected flag asset},
    pass = {All generated pixel coordinates match their expected values},
    result = \textcolor{Green}{PASS}
}
\section{Nonfunctional Requirements Evaluation}
The following are the test cases that were evaluated for testing the
non-functional requirements of this system.

\pagebreak
\subsection{Look and Feel Requirements}
\testmanual{
    name = {LF-01},
    desc =  Tests that the user interface makes it easy for users to navigate the program
    using developer judgement and consensus,
    type = {Manual, Dynamic},
    user = {Developers},
    pass = {Average developer consensus of at least $\hyperlink{theta_max}{\Theta_{max}}$},
    result = \textcolor{Green}{PASS} with an agreement of 100\%
}
\testmanual{
    name = {LF-02},
    desc =  Tests that the image quality and aesthetics of the generated flags are
    aesthetically pleasing using a rating survey with options on a scale of 1-10,
    type = {Manual, Functional, Dynamic},
    user = {Testers},
    pass = {Average survey score of at least $\hyperlink{theta_mid}{\Theta_{mid}}$},
    result = \textcolor{Green}{PASS} with an agreement of 95\%
}
\testmanual{
    name = {LF-03},
    desc =  Tests that the colours used in the generated flag are within certain colour
    ranges using developer judgement and consensus,
    type = {Manual, Functional, Dynamic},
    user = {Developers},
    pass = {Average developer consensus of at least $\hyperlink{theta_max}{\Theta_{max}}$},
    result = \textcolor{Green}{PASS} with an agreement of 90\%
}
\testmanual{
    name = {LF-04},
    desc =  Tests that all flag components are visible in the generated flag image using
    developer judgement and consensus,
    type = {Manual, Functional, Dynamic},
    user = {Developers},
    pass = {Average developer consensus of at least $\hyperlink{theta_min}{\Theta_{min}}$},
    result = \textcolor{Green}{PASS} with an agreement of 100\%
}

\pagebreak
\subsection{Usability and Humanity Requirements}
\testmanual{
    name = {UH-01},
    desc =  Tests that user interface components are placed in a logically flowing manner
    using developer judgement and consensus,
    type = {Manual, Dynamic},
    user = {Developers},
    pass = {Average developer consensus of at least $\hyperlink{theta_max}{\Theta_{max}}$},
    result = \textcolor{Green}{PASS} with an agreement of 95\%
}
\testmanual{
    name = {UH-02},
    desc =  Tests that the user must not have to jump between interfaces to accomplish a task
    using developer judgement and consensus,
    type = {Manual, Functional, Dynamic},
    user = {Developers},
    pass = {Average developer consensus of at least $\hyperlink{theta_max}{\Theta_{max}}$},
    result = \textcolor{Green}{PASS} with an agreement of 100\%
}
\testmanual{
    name = {UH-03},
    desc =  Tests that the program is easy-to-use for people aged
    $\hyperlink{min_age}{MIN\_AGE}$ or older using tester judgement and consensus,
    type = {Manual, Dynamic},
    user = {Testers},
    pass = {Average tester consensus of at least $\hyperlink{theta_max}{\Theta_{max}}$},
    result = \textcolor{Green}{PASS} with an agreement of 100\%
}
\testmanual{
    name = {UH-05},
    desc =  Tests that user is able to select output specifications (type of hashing\, type
    of image file\, etc.) using developer judgement and consensus,
    type = {Manual, Functional, Dynamic},
    user = {Developers},
    pass = {Average developer consensus of at least $\hyperlink{theta_max}{\Theta_{max}}$},
    result = \textcolor{Green}{PASS} with an agreement of 90\%
}
\testmanual{
    name = {UH-07},
    desc =  Tests that the user is able to use the program with no prior experience using developer and tester judgement and consensus,
    type = {Manual, Dynamic},
    user = {Developers, Testers},
    pass = {Average developer/tester consensus of at least
    $\hyperlink{theta_max}{\Theta_{max}}$},
    result = \textcolor{Green}{PASS} with an agreement of 90\%
}
\testmanual{
    name = {UH-08},
    desc =  Tests that the user is able to access a brief instructions blurb using tester
    judgement and consensus,
    type = {Manual, Dynamic},
    user = {Testers},
    pass = {Average tester consensus of at least $\hyperlink{theta_max}{\Theta_{max}}$},
    result = \textcolor{Green}{PASS} with an agreement of 100\%
}
\testmanual{
    name = {UH-09},
    desc =  Tests that the user is able to use the program by following the main user interface
    instructions only using developer and tester judgement and consensus,
    type = {Manual, Dynamic},
    user = {Developers, Testers},
    pass = {Average developer/tester consensus of at least
    $\hyperlink{theta_max}{\Theta_{max}}$},
    result = \textcolor{Green}{PASS} with an agreement of 100\%
}
\testmanual{
    name = {UH-10},
    desc =  Tests that the program uses consistent language throughout using developer
    judgement and consensus,
    type = {Manual, Dynamic},
    user = {Developers},
    pass = {Average developer consensus of at least $\hyperlink{theta_max}{\Theta_{max}}$},
    result = \textcolor{Green}{PASS} with an agreement of 100\%
}
\testmanual{
    name = {UH-11},
    desc =  Tests that the program uses simplified terminology wherever possible using
    developer judgement and consensus,
    type = {Manual, Dynamic},
    user = {Developers},
    pass = {Average developer consensus of at least $\hyperlink{theta_max}{\Theta_{max}}$},
    result = \textcolor{Green}{PASS} with an agreement of 100\%
}
\testmanual{
    name = {UH-12},
    desc =  Tests that the program uses easy-to-read fonts and font sizes using tester
    judgement and consensus,
    type = {Manual, Dynamic},
    user = {Testers},
    pass = {Average tester consensus of at least $\hyperlink{theta_max}{\Theta_{max}}$},
    result = \textcolor{Green}{PASS} with an agreement of 95\%
}

\subsection{Performance Requirements}
\testmanual{
    name = {PE-01-02},
    desc =  Tests that the program minimizes the time taken to generate and process the downloading of an image using developer judgement and consensus,
    type = {Manual, Functional, Dynamic},
    user = {Developers},
    pass = {Average developer consensus of at least $\hyperlink{theta_max}{\Theta_{max}}$},
    result = \textcolor{Green}{PASS} with an agreement of 100\%
}
\testmanual{
    name = {PE-03},
    desc =  Tests that the program minimizes the time taken to load in a user’s gallery (i.e.
    average gallery load time is below $\hyperlink{max_load_time}{MAX\_LOAD\_TIME}$) using
    timing tests and developer judgement and consensus,
    type = {Manual, Functional, Dynamic},
    user = {Developers},
    pass = {Program loads in the user's gallery with an average load time less than
    $\hyperlink{max_load_time}{MAX\_LOAD\_TIME}$ and an average developer consensus of at
    least $\hyperlink{theta_max}{\Theta_{max}}$},
    result = \textcolor{Green}{PASS} with an average load time within
    $\hyperlink{max_load_time}{MAX\_LOAD\_TIME}$ and an agreement of 100\%
}
\testmanual{
    name = {PE-04},
    desc =  Tests that the different hashing systems all deliver consistent and precise
    outputs using developer judgement and consensus,
    type = {Manual, Functional, Dynamic},
    user = {Developers},
    pass = {Average developer consensus of at least $\hyperlink{theta_max}{\Theta_{max}}$},
    result = \textcolor{Green}{PASS} with an agreement of 100\%
}
\testmanual{
    name = {PE-05},
    desc =  Tests that the generated images all have accurately placed components (i.e.
    matching the templates) using developer judgement and consensus,
    type = {Manual, Functional, Dynamic},
    user = {Developers},
    pass = {Average developer consensus of at least $\hyperlink{theta_max}{\Theta_{max}}$},
    result = \textcolor{Green}{PASS} with an agreement of 100\%
}
\testmanual{
    name = {PE-06},
    desc =  Tests that the colours in the generated flag image are precise (i.e. the
    hexadecimal value produces the correct colour) using developer judgement and consensus,
    type = {Manual, Functional, Dynamic},
    user = {Developers},
    pass = {Average developer consensus of at least $\hyperlink{theta_max}{\Theta_{max}}$},
    result = \textcolor{Green}{PASS} with an agreement of 100\%
}
\testmanual{
    name = {PE-07},
    desc =  Tests that the program is available to run anytime in the day using developer
    judgement and consensus,
    type = {Manual, Dynamic},
    user = {Developers},
    pass = {Average developer consensus of at least $\hyperlink{theta_max}{\Theta_{max}}$},
    result = \textcolor{Green}{PASS} with an agreement of 100\%
}
\testmanual{
    name = {PE-09},
    desc =  Tests that the program limits the number of users to one per machine using
    developer judgement and consensus,
    type = {Manual, Functional, Dynamic},
    user = {Developers},
    pass = {Average developer consensus of at least $\hyperlink{theta_max}{\Theta_{max}}$},
    result = \textcolor{Green}{PASS} with an agreement of 100\%
}
\testmanual{
    name = {PE-10},
    desc =  Tests that the program allows for the addition of other hashing functions
    (relatively easily) using developer judgement and consensus,
    type = {Manual, Functional, Dynamic},
    user = {Developers},
    pass = {Average developer consensus of at least $\hyperlink{theta_max}{\Theta_{max}}$},
    result = \textcolor{Green}{PASS} with an agreement of 100\%
}
\testmanual{
    name = {PE-11},
    desc =  Tests that the program allows for the addition of other flag components
    (relatively easily) using developer judgement and consensus,
    type = {Manual, Functional, Dynamic},
    user = {Developers},
    pass = {Average developer consensus of at least $\hyperlink{theta_max}{\Theta_{max}}$},
    result = \textcolor{Green}{PASS} with an agreement of 95\%
}

\subsection{Operational and Environmental Requirements}
\testmanual{
    name = {PE-13},
    desc =  Tests that the program does not require an internet connection to function using
    developer judgement and consensus,
    type = {Manual, Functional, Dynamic},
    user = {Developers},
    pass = {Average developer consensus of at least $\hyperlink{theta_max}{\Theta_{max}}$},
    result = \textcolor{Green}{PASS} with an agreement of 100\%
}
\testmanual{
    name = {PE-14},
    desc =  Tests that the program can run on any computer that can support the Python
    language using developer judgement and consensus,
    type = {Manual, Functional, Dynamic},
    user = {Developers},
    pass = {Average developer consensus of at least $\hyperlink{theta_max}{\Theta_{max}}$},
    result = \textcolor{Green}{PASS} with an agreement of 100\%
}

\subsection{Interfacing with Adjacent Systems Requirements}
\testmanual{
    name = {PE-15},
    desc =  Tests that the program does not alter files outside the working directory using
    developer judgement and consensus,
    type = {Manual, Functional, Dynamic},
    user = {Developers},
    pass = {Average developer consensus of at least $\hyperlink{theta_max}{\Theta_{max}}$},
    result = \textcolor{Green}{PASS} with an agreement of 100\%
}
	
\section{Comparison to Existing Implementation}
The original source project, PAGAN only had about 15 test cases spread
throughout the module files, and did not do a thorough job testing the
program. There was no additional documentation provided in the source code
with any reference to tracing these tests to specific requirements. For Random
Flag Generator multiple documents were produced that help trace between
requirements and testing. These documents include: a software requirements
specification, a test plan, and additional design documentation. In all, with
the added traceability through the documentation, over 60 test cases were
developed and performed for automated functional testing, and multiple
non-functional requirements tests were also performed to ensure program
robustness and correctness before finalizing the project.

\section{Unit Testing}
Unit testing was a large part of the testing done for Random Flag Generator.
All functional modules had unit tests written and performed. Unit testing is
crucial to this project, as there are many small specific functions that
perform calculations or make deicisons that get used by other modules. All
modules must produce the correct output to be taken in by the next module as
input, to ensure a correct and complete final generated image. If at any
point, any of the modules' functionality is incorrect, the final flag
generation cannot be guaranteed. Additionally, given the random nature of the
program(ie. decisions using hashing), it was important to make sure most if
not all of the possible options were able to be selected, ensuring the widest
array of flag options possible.

\section{Changes Due to Testing}
No major changes occurred due to the testing that took place. Small changes
occured in individual modules to better handle boundary cases. For example,
the Hash Generator was not able to handle certain hashing algorithm type
inputs like SHAKE, as it required additional user input. A workaround was
implemented for this algorithm type, when this bug arose during testing.

\section{Automated Testing}
All of the non-GUI functionality of the program was tested automatically using
the pytest unit test framework. A suite of unit tests were developed and then
ran for each functional module. The testing of the GUI was not automated, as
ad-hoc user input was better for GUI testing. Additionally, image generation
(and correctness) were able to be tested using pixel by pixel comparisons
(done in a unit test format).

\section{Trace to Requirements}
Traceability matrices documenting the traceability between the test cases and
the various functional and non-functional requirements of this system are
shown on the next few pages.

\begin{landscape}

\begin{table}[h!]
\centering
\caption{\textbf{Traceability Matrix for Functional Requirements I}}
\label{tab:trace_matrix_01}
\begin{tabular}{|c|c|c|c|c|c|c|c|c|c|c|c|c|c|c|c|}
\hline
& \multicolumn{14}{c|}{Requirements} \\
\hline
Test Cases & FR1 & FR2 & FR3 & FR4 & FR5 & FR6 & FR7 & FR8 & FR9 & FR10 & FR11 & FR12 & FR13 & FR14 \\
\hline
FR-01 & X & & & & & & & & & & & & & \\
\hline
FR-02 & & X & & & & & & & & & & & & \\
\hline
FR-03 & & & X & & & & & & & & & & & \\
\hline
FR-04-08 & & & & X & & & & X & & & & & & \\
\hline
FR-05-09 & & & & & X & & & & X & & & & & \\
\hline
FR-06-10 & & & & & & X & & & & X & & & & \\
\hline
FR-07-11 & & & & & & & X & & & & X & & & \\
\hline
FR-12-13 & & & & & & & & & & & & X & X & \\
\hline
FR-14 & & & & & & & & & & & & & & X \\
\hline
\end{tabular}
\end{table}

\newpage

\begin{table}[h!]
\centering
\caption{\textbf{Traceability Matrix for Functional Requirements II/Non-Functional Requirements I}}
\label{tab:trace_matrix_02}
\begin{tabular}{|c|c|c|c|c|c|c|c|c|c|c|c|c|}
\hline
& \multicolumn{12}{c|}{Requirements} \\
\hline
Test Cases & FR15 & FR16 & FR17 & FR19 & LF1 & LF2 & LF3 & LF4 & UH1 & UH2 & UH3 & UH5 \\
\hline
FR-15 & X & & & & & & & & & & & \\
\hline
FR-16 & & X & & & & & & & & & & \\
\hline
FR-17 & & & X & & & & & & & & & \\
\hline
FR-19 & & & & X & & & & & & & & \\
\hline
LF-01 & & & & & X & & & & & & & \\
\hline
LF-02 & & & & & & X & & & & & & \\
\hline
LF-03 & & & & & & & X & & & & & \\
\hline
LF-04 & & & & & & & & X & & & & \\
\hline
UH-01 & & & & & & & & & X & & & \\
\hline
UH-02 & & & & & & & & & & X & & \\
\hline
UH-03 & & & & & & & & & & & X & \\
\hline
UH-05 & & & & & & & & & & & & X \\
\hline
\end{tabular}
\end{table}

\newpage

\begin{table}[h!]
\centering
\caption{\textbf{Traceability Matrix for Non-Functional Requirements II}}
\label{tab:trace_matrix_03}
\begin{tabular}{|c|c|c|c|c|c|c|c|c|c|c|c|c|}
\hline
& \multicolumn{12}{c|}{Requirements} \\
\hline
Test Cases & UH7 & UH8 & UH9 & UH10 & UH11 & UH12 & PE1 & PE2 & PE3 & PE4 & PE5 & PE6 \\
\hline
UH-07 & X & & & & & & & & & & & \\
\hline
UH-08 & & X & & & & & & & & & & \\
\hline
UH-09 & & & X & & & & & & & & & \\
\hline
UH-10 & & & & X & & & & & & & & \\
\hline
UH-11 & & & & & X & & & & & & & \\
\hline
UH-12 & & & & & & X & & & & & & \\
\hline
PE-01-02 & & & & & & & X & X & & & & \\
\hline
PE-03 & & & & & & & & & X & & & \\
\hline
PE-04 & & & & & & & & & & X & & \\
\hline
PE-05 & & & & & & & & & & & X & \\
\hline
PE-06 & & & & & & & & & & & & X \\
\hline
\end{tabular}
\end{table}

\newpage

\begin{table}[h!]
\centering
\caption{\textbf{Traceability Matrix for Non-Functional Requirements III}}
\label{tab:trace_matrix_03}
\begin{tabular}{|c|c|c|c|c|c|c|c|}
\hline
& \multicolumn{7}{c|}{Requirements} \\
\hline
Test Cases & PE7 & PE9 & PE10 & PE11 & PE13 & PE14 & PE15 \\
\hline
PE-07 & X & & & & & & \\
\hline
PE-09 & & X & & & & & \\
\hline
PE-10 & & & X & & & & \\
\hline
PE-11 & & & & X & & & \\
\hline
PE-13 & & & & & X & & \\
\hline
PE-14 & & & & & & X & \\
\hline
PE-15 & & & & & & & X \\
\hline
\end{tabular}
\end{table}

\end{landscape}

\section{Trace to Modules}		
The traceability table documenting the traceability between the functional
requirements and the system module(s) that implement them is shown below.

\begin{table}[h!]
\centering
\begin{tabular}{p{0.2\textwidth} p{0.6\textwidth}}
\toprule
\textbf{Req.} & \textbf{Modules}\\
\midrule
FR1 & \mref{M7}\\
FR2 & \mref{M7}\\
FR3 & \mref{M7}\\
FR4 & \mref{M1}, \mref{M2}, \mref{M3}, \mref{M4}, \mref{M5}, \mref{M6}, \mref{M7}\\
FR5 & \mref{M8}\\
FR6 & \mref{M1}, \mref{M2}, \mref{M3}, \mref{M4}, \mref{M5}, \mref{M6}, \mref{M11}\\
FR7 & \mref{M6}\\
FR8 & \mref{M1}, \mref{M2}, \mref{M3}, \mref{M4}, \mref{M5}, \mref{M6}, \mref{M7}\\
FR9 & \mref{M8}\\
FR10 & \mref{M1}, \mref{M2}, \mref{M3}, \mref{M4}, \mref{M5}, \mref{M6}, \mref{M11}\\
FR11 & \mref{M6}, \mref{M7}, \mref{M9}\\
FR12 & \mref{M10}\\
FR13 & \mref{M10}\\
FR14 & \mref{M11}\\
FR15 & \mref{M11}\\
FR16 & \mref{M11}\\
FR17 & \mref{M9}\\
FR19 & \mref{M9}\\
\bottomrule
\end{tabular}
\caption{Trace Between Requirements and Modules}
\label{TblRT}
\end{table}

\section{Code Coverage Metrics}

\subsection{Symbolic Parameters}
The definition of the test cases will call for SYMBOLIC\_CONSTANTS. Their values are defined in this section for easy maintenance. \\ \\
$\hypertarget{min_age}{MIN\_AGE}$ = 7 \\
$\hypertarget{max_find_time}{MAX\_FIND\_TIME}$ = 5 \\
$\hypertarget{max_load_time}{MAX\_LOAD\_TIME}$ = 2 \\
$\hypertarget{theta_min}{\Theta_{min}}$ = 80 \\
$\hypertarget{theta_mid}{\Theta_{mid}}$ = 90 \\
$\hypertarget{theta_max}{\Theta_{max}}$ = 100 \\

\bibliographystyle{plainnat}

\bibliography{SRS}

\end{document}